\chapter*{Intellectual Property}
\addcontentsline{toc}{chapter}{Intellectual Property}
The candidate confirms that the work submitted is his/her own and that appropriate credit has been given where reference has been made to the work of others.

This copy has been supplied on the understanding that it is copyright material and that no quotation from the thesis may be published without proper acknowledgement.

© \submissionyear\ The University of Leeds, \name

\vspace{2cm}
Signed 
\makebox[4cm][c]{\raisebox{-2ex}{\includegraphics[width=3cm, height=1cm]{example-image}}} % replace with your signature

\chapter*{Acknowledgements}
\addcontentsline{toc}{chapter}{Acknowledgements}
This research has been carried out by a team which has included (name the individuals). My own contributions, fully and explicitly indicated in the thesis, have been......(please specify)” The other members of the group and their contributions have been as follows: (please specify).

\chapter*{Abstract}
\addcontentsline{toc}{chapter}{Abstract}

At present, tree populations worldwide face unprecedented threats from invasive pests and pathogens. In turn, poor tree health endangers ecological function, timber production and human wellbeing. 
From first principles, this thesis incrementally extends a simple percolation model of forest-based epidemics into a more involved stochastic dispersal framework combined with host data. 
The research aims to construct robust epidemic models of tree disease over the landscape of Great Britain and, afterwards, optimise a novel epidemic control strategy.

The approach developed here couples two spatially-explicit epidemic models at different scales. 
First, a non-local stochastic model of pathogen dispersal between trees is constructed. 
Then, the small-scale epidemic model is projected onto a large-scale distribution of host abundance, resulting in an $R_0$-map across Great Britain. 
Subsequently, a clustering algorithm is employed to identify high-risk regions in the $R_0$-map. 
Initial results indicate a global epidemic phase transition across the distribution, conditional on an infectivity parameter.
The approach to `spatially scale' an epidemic model over the entire landscape is computationally efficient, flexible and adaptable to many pests and pathogens. 

Numerous studies have sought to understand and optimise epidemic control in botanical populations. 
The mainstream control paradigm generally seeks to optimise an `eradication radius' about infected symptomatic trees over a relatively small spatial scale. However, large-scale epidemic control based solely on the spatial distribution of hosts has yet to be explored in depth. 
As such, this thesis will also examine how host heterogeneity, combined with targeted epidemic control, can give rise to natural `pinch-points' that may bottleneck the epidemic spread between regions. 

Ultimately, this investigation intends to help policymakers reach informed decisions about \textit{where} to focus control in the landscape.



\newcommand{\RNum}[1]{\uppercase\expandafter{\romannumeral #1\relax}}
