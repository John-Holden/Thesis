\chapter*{Intellectual Property}
\addcontentsline{toc}{chapter}{Intellectual Property}
The candidate confirms that the work submitted is his/her own and that appropriate credit has been given where reference has been made to the work of others.

This copy has been supplied on the understanding that it is copyright material and that no quotation from the thesis may be published without proper acknowledgement.

© \submissionyear\ The University of Leeds, \name

\vspace{2cm}
Signed 
\makebox[4cm][c]{\raisebox{-2ex}{\includegraphics[width=3cm, height=1cm]{example-image}}} % replace with your signature

\chapter*{Acknowledgements}
\addcontentsline{toc}{chapter}{Acknowledgements}
Firstly, I would like to thank my primary supervisor Dr James Smith for the support, guidance, great company (particularly in Brazil) and everything he taught me along the way. 
My gratitude also extends to my co-survivors in Newcastle, Dr Nick Parker and Dr Andrew Baggeley, for all their support and insight. 
In particular, I'm deeply grateful for being hosted in Newcastle during the first year of the project.
Similarly, I want to thank my co-supervisors in Leeds, Dr Melvin Holmes and Dr Rammile Ettelaie, for their help, support and encouragement. 

I thank Dr Sam Grant for introducing me to the role of policymaking in tree health and for helpful discussions. I'm also thankful for the generous funding made available from DEFRA. 
Additionally, I extend many thanks to Dr Siro Orozco-Fuentes, who gave this project a firm footing during the initial phases.

I'm forever indebted to my friends and family for giving reason to the madness and hardship. 
I offer sincere thanks to Marcin Kupilas for being a magnificent friend and an influential part of this PhD adventure; his curiosity and endless desire to learn inspired me throughout the project.
Moreover, I'm deeply grateful to my good friend, Bradley Lister, for being there throughout the tough times. Brad's unconditional support, ongoing encouragement, and positive energy kept me sane during the lockdown. 
Likewise, I'd like to thank my cousin Luke for always being there and offering perspective, a positive mindset and support when times went beyond challenging.
And lastly, I profoundly thank my Mum and Dad and Sister for their undying support and always believing in me. Without them, this journey would not be possible.


\chapter*{Abstract}
\addcontentsline{toc}{chapter}{Abstract}

Presently, tree populations worldwide face unprecedented threats from invasive pests and pathogens endangering biodiversity, timber production and human wellbeing. 
From first principles, this thesis incrementally extends a simple percolation model of forest-based epidemics into a more involved stochastic dispersal framework combined with tree canopy data. 
The approach developed here couples two spatially-explicit epidemic models at different scales. 
First, a non-local stochastic model of pathogen dispersal between trees is constructed. 
Second, the small-scale epidemic model is projected onto a large-scale distribution of host abundance, resulting in an $R_0$-map across Great Britain. 
Subsequently, a clustering algorithm is employed to identify high-risk regions in the $R_0$-map. 
Initial results indicate a global epidemic phase transition across the distribution, conditional on an infectivity parameter.
The approach to `spatially scale-up' an epidemic model over the entire landscape is computationally efficient, flexible and adaptable to many pests and pathogens. 
In addition, numerous studies have sought to understand and optimise epidemic control in botanical populations. 
The mainstream control paradigm generally seeks to optimise an `eradication radius' about infected symptomatic trees over a relatively small spatial scale. However, large-scale epidemic control based solely on the spatial distribution of hosts has yet to be explored in depth. 
As such, this thesis will also examine how host heterogeneity, combined with targeted epidemic control, can give rise to natural `pinch-points' that may slow the epidemic spread between regions. 
Ultimately, this investigation intends to help policymakers reach informed decisions about where to focus control in the landscape of Great Britain.

\newcommand{\RNum}[1]{\uppercase\expandafter{\romannumeral #1\relax}}
