\chapter{Regionally Containing Epidemics: Modelling Ash dieback}

% paper title: Large-scale control based on regional containment
% there is a surprising lack of, simplified, ash dieback models in the literature...\ciations... <-- double, triple and quadruple check!
% without host demography 
% landscape level control strategy

Previously in Chapter \ref{ch5:dispersal-model}, we considered a generic $SIR$ model that spread via wind-borne dispersal. Constructing the dispersal model resolved the major problem witnessed in Chapter \ref{fig:ch4_uk_spread}, namely, the failure of the SLM to spread on a realistic host density. However, the findings of Chapter \ref{ch5:dispersal-model} lacked biological specificity and thus, came short of aiding plant health. This chapter aims to address this. The present chapter will construct a simplified model of ash dieback, based on the dispersal model, and be utilised to develop an epidemic control strategy.

There has been a great deal of work carried out into the nature of control in plant and tree-based epidemics\footnote{See section \ref{chapter2:plant-ecologoy} for a review on the control in plant-based epidemics.}. In particular, the spatial structure of plant-hosts is an essential factor when considering how to manage an outbreak \cite{spatial-control-optimisation, control-heterogeneous-landscapes}. The accepted paradigm of control typically considers infected tree removals over a relatively small spatial scale, near infected hosts \cite{WEBIDEMICS}, or more broadly, ahead of the wavefront \cite{large-scale-control}. However, landscape-level epidemic control, based solely on the structure of large-scale spatial distribution of hosts incorporating topography, has yet to be explored in-depth.\\

As such, in this chapter, we will examine how host-heterogeneity, under the influence of a wind-dispersed pathogen, can give rise to natural pinch-points and fault lines in the spatial distribution of hosts. Population pinch points may give rise to a bottleneck in the epidemic spread, which in principle, may be exploited with targeted tree felling to fragment the host population with minimised effort. In essence, a strategy of 'regional containment', targeting the local wind-based pathogen dispersal mechanism, is formulated and scaled up over large spatial scales. Similar concepts for crop and livestock diseases have been outlined \cite{PAPAIX201435, GILIOLI20131, Gilligan-disease-management}, however, to our knowledge, this has not been generalised to tree population distributions over large spatial scales.

A simplified $SEIR$-type model of ash dieback is developed to demonstrate this control strategy, alongside an appropriate definition of the reproductive ratio\textemdash denoted by $R_0$. The value of $R_0$ is projected onto the map of ash tree canopy cover in GB, as given by \cite{hill.data}. From the $R_0$ maps we construct, a simple notion of epidemiological connectivity can be defined and visualised through a susceptible '$R_0$-cluster' over the population of GB ash trees. This leads us to develop a heuristically-based fragmentation algorithm. As we define it, fragmentation considers which locations in the population, if artificially taken below $R_0 = 1$ through felling, would disrupt epidemiological connectivity\textemdash, thus leading to containment. Epidemic containment in the largest $R_0$-cluster is then analysed and shown to be most applicable over a specific range of infectivity parameters.

It is widely accepted that ADB will wipe out the vast majority of ash in Great Britain over the next few decades \cite{ash-dieback-costs}. Therefore, large-scale control efforts aim to slow the spread, in contrast, to complete containment. % Find references of current efforts/guidelines 
A slower rate of spread benefits ash populations allowing them to recover alongside artificial replanting.% reference 
Although the strategy of control presented in this chapter is demonstrated on a simplifed model of ADB, the results are generic and could be applied to any wind-dispersed pathogen.
The pathosystem ADB presents an interesting and relevant case study of an emerging epidemic whose reproductive mode is both seasonal and subject to LDD. The]refore, a model of ADB brings together the several key elements of chapter \ref{ch5:dispersal-model}, measuring $R_0$ over different temporal and spatial scales. 

The challenge of controlling ADB primarily reflects the challenge of containing a pathogen that spreads via long-distance dispersal (LDD). As such, developing a landscape-level control strategy when there is LDD (and epidemic uncertainty) present several obstacles that must be acknowledged beforehand. Firstly, the complexity of modelling ADB is a thesis in and of itself, few parameters are known, spatial and genetic variations are significant \cite{stocks2017first, mckinney2014ash} and a dependency on the landscape \cite{doi:10.1111/1365-2745.13383}. Moreover, HP can infect ash through diverse mechanisms such as water-course and contaminated soil and LDD means that new and distant foci can emerge over large distances without the need for nearby ash\textemdash for a more detailed review of ADB, including the challenges of control and biology, see chapter \ref{chapter2:litrevieiw}.

% multi-scale approaches have been outlined \cite{hart2020theoretical}
% multi-seasonal frameworks comprise a common theme in the spread of crop-based epidemics see  x, y, and typically involve soil-borne nematodes-based outbreaks \cite{tankam2020modelling} <- see references inside.
% ash dieback has a strong morality rate \cite{stocks2017first}

\section{Constructing an $SEIR$ model}

\subsection{On the biology of ash dieback}

The primary infection mechanism occurs when the fungal spores of \textit{H.pseudoalbidus} (HP) are wind-dispersed in the summer and land on the leaves of susceptible ash. Once the pathogen colonises a leaf, it spreads to the xylem and then throughout the whole tree. % references on mechanism
An infected tree will then shed its leaves in the autumn. Fungal fruiting bodies then grow on dead leaf litter until summertime; at that point, spores produced by the fruiting body are wind-dispersed and continue the cycle by producing new secondary infections.% expand on the fruiting body spore production.

Specific to ADB, there are two stages of dispersal: the causal agent HP is dispersed on infected leaves during the yearly shed of ash, secondly, wind dispersal of fungal spores. Fortunately, the aggregate behaviour of two-stage dispersal is captured by sampling the amount of fungal spores with distance from a nearby source of infected ash \cite{grosdidier2018tracking}\textemdash discussed more below. 
Be that as it may, an infected tree is treated as the dispersal site in the model, although the realistic description is not so simple.

The life cycle can be understood to have two phases of growth, sexual and asexual. % reference
The asexual phase occurs when the pathogen infects and amplifies through the host \textcolor{red}{and occurs all year round}. The sexual phase of HP occurs during the summer months, from June until September when wind-dispersed spores infect new ash trees. % double check the red-highlighted

The seasonal infection cycle of ADB resembles that of crop-based disease \cite{tankam2020modelling}. Although, in the context of crop disease, removal usually coincides with harvest time. In contrast, ash infected with HP can be removed by the pathogen over a time frame spanning years. Once infected, the time ash survive depends on a plethora of factors such as age, surrounding landscape, genetic susceptibility or the particular mode of pathogen infection\footnote{For example, the pathogen can colonise the root-system \cite{schumacher2011general}, usually in severely infected ash \cite{https://doi.org/10.1111/mpp.12073}. From this point, it is only a matter of time before opportunistic fungi invade and significantly accelerate mortality \cite{enderle2013temporal}.}. In Germany, a forest stand of planted ash trees had a $73\%$ mortality rate after five years \cite{langer2015ash} (as cited in a review \cite{enderle2017ash}), while observations of ADB progression in Austria suggest a low mortality rate of $5\%$ measured over a two-year window \cite{kessler2012dieback}. Furthermore, a study conducted at different sites throughout England suggests a time-scale ranging between $3-15$ years of infected tree growth before death \cite{wylder2018evidence}.

% The mortality of infected ash infected with HP is high, recent estimates suggest around $95\%$ \cite{stocks2017first}. % check \cite{ash-dieback-costs} paper for other estimates

\subsection{Infection dynamics}

\begin{table}[h]
\centering
\begin{tabular}{l l l}
\hline
\textbf{Model parameter} & \textbf{Description} & \textbf{Value(s) taken}\\
\hline
$\rho$  & Tree density & $0.00 - 0.10$ \\ 
$\beta$ & Infectivity & $0.00010 - 0.00100$ \\
$\ell_{ga}$ & Gaussian dispersal scale parameter& $196\mathrm{m}$ \\
$\ell_{pl}$ & Power-law dispersal scale parameter& $203\mathrm{m}$ \\
$a$ & Power-law dispersal exponent & $3.3$ \\
$T$ & Sporulation peak & June - September \\
$t$ & Time-step & $1\ \mathrm{day}$\\
$R_0$ & Mean reproduction number & $0-20$ \\
$\mathcal{L}$ & Lattice dimension & $1000\times1000$ \\
$\mathcal{A}$ & Domain area & $5\mathrm{km}\times5\mathrm{km}$ \\
$\gamma$ & Lattice constant & $5\mathrm{m}$ \\
$\mu$ & Peak leaf-shedding of ash & November \\
$\sigma$ & Leaf-shedding standard deviations & 2 weeks \\
$\lambda$ & Mean exponentially-distributed infectious life-time  & $5 \mathrm{years}$ \\
\hline
\end{tabular}
\caption{Parameters used in the $SEIR$ model of ash dieback. The dispersal parameters are taken from \cite{grosdidier2018tracking} and the typical tree densities of ash are informed from by \cite{hill.data}.}
\label{tab:SEIR-model}
\end{table}

As before, the hosts' distribution, in this case ash, is initialised by a Bernoulli trial with probability $\rho$ according to a binomial distribution\textemdash thus giving rise to a flat and randomly distributed landscape of trees. The probability $\rho$ defines the host density inside a square domain of size $\mathcal{L}$. Each lattice point is chosen to represent a $5\mathrm{m}\times5\mathrm{m}$ patch of land that approximates the canopy cover of an ash tree, this resolution yields an upper bound of $400$ ash trees per hectare of canopy cover \cite{ash-tree2, ash-tree1}.

However, this time, we include an extra latently infected (or exposed) compartment, denoted by $E$, into the model. Thus, a tree can be in one of four compartments and transition through $S\rightarrow E \rightarrow I \rightarrow R$, without the possibility of recovery. Figure \ref{fig:SEIR-transitions} shows a typical scenario; an infected ash tree in the $n^{th}$ cycle, or equivalently $n$ years after the initial outbreak, may infect ash in the $S$ compartment. Newly infected ash will transition into the $n^{th}$ $E$ compartment, denoted by $E_n$, and become infectious in the following year $I_{n+1}$. 

The compartments $(E_1, E_n,..,E_n)$ do not represent different biological states, the index $n$ is included for convenience to highlight the fact that latently infected ash take, on average, one year before producing new secondary infections.

\begin{figure}
    \centering
    \includegraphics[scale=0.30]{chapter6/figures/fig1.pdf}
    \caption{The $SEIR$ model of ash dieback. In year $n$ of an outbreak, an infectious tree may cause a transition $S\rightarrow E_n$, depicted by the bottom dashed grey arrow. A tree that becomes latently infected in year $n$ will lead to an infectious tree in year $n+1$. Eventually, all infected ash are removed without the possibility of recovery.}
    \label{fig:SEIR-transitions}
\end{figure}

As define here, an ash tree transitions into the infectious compartment when leaf shedding begins in Autumn. A Gaussian distribution centred in November with a two-week standard deviation represents this in the model. This particular choice of standard deviation, although best-estimated, is ultimately ill-informed. However, because infected ash do not produce any secondary infections until the following summer, we can afford some degree of flexibility modelling the exact timing of transitions between $E_{n}\rightarrow I_{n+1}$; this flexibility is demonstrated in-depth later on.

Once in the infectious compartment, an infected ash tree will not give rise to any secondary infections until summer when fruiting bodies produce their spores on dead leaf litter. Thus, infectivity is now time-dependent and can be captured by a 'sporulation' function $\phi(t)$. Sporulation functions have been investigated in the context of a time-varying infectivity parameter \cite{time-varying-infectivity}. In the case of ADB, the function $\phi(t)$ is used to mirror the life cycle of ADB where new secondary infections only arise during the summertime sporulation \cite{https://doi.org/10.1111/mpp.12073}. That is, non-zero from June until September, and zero otherwise\footnote{Variations in ADB sporulation have been noted between European countries \cite{https://doi.org/10.1111/mpp.12073}, along with the potential for early-onset sporulation in the face of favourable environmental conditions. Although, the most generally agreed upon sporulation period is thought to be from June to September.}.

To model dispersal, a thin-tailed Gaussian was considered alongside a fat-tailed inverse power-law distribution\textemdash see table \ref{tab:SEIR-model}. As seen previously in chapter \ref{ch5:dispersal-model}, a probability of transition $S_x \rightarrow E_x$ can be seen to follow:
\begin{equation}
    Pr(S_{x} \rightarrow E_{x} ;\ I_{x^{\prime}} ) = \beta  \phi(T) \exp\Big[\frac{-r^2}{2\ell^2}\Big] 
\end{equation}
\begin{equation}
    Pr(S_{x} \rightarrow E_{x} ;\ I_{x^{\prime}} ) = \beta \phi(T) (1 - r/\ell)^{-r}
\end{equation}
where $\phi(t)$ is a time-dependant function reflecting the seasonal life-cycle of ADB (the precise functional form is discussed in-depth later) and $r$ is the distance between $S_x$ and $I_{x^\prime}$. Thus, the inclusion of an exposed category, together with a sporulation function, constrain the dynamics of ADB whereby: A) infected ash may display symptoms but crucially not disperse any infectious material B) ash may shed infected leaves in Autumn/winter, and thereby disperse infected material, but not lead to any secondary infections until the following summer.

Finally, the last transition to consider is from infected to removed $I_{n}\rightarrow R$. Given a $95\%$ mortality rate, ADB can be regarded as lethal. Therefore, once an ash tree becomes infected, an eventual transition to the $R$ compartment is assumed with a probability of one\footnote{Interestingly, edge-cases can contradict this assumption. For example, a sufficiently large quantity of inoculum deposited on ash leaves can result in leaf shed before the infection has the chance to spread throughout the tree \cite{https://doi.org/10.1111/mpp.12073}.}. As a first approximation, infected ash were chosen to have exponentially distributed life-times with a mean of five years, see table \ref{tab:SEIR-model}. 

The precise probability distribution describing $I_{n}\rightarrow R$ is, to my knowledge, non-existent in the literature. Although, observations of mortality ratios after years of infection provided some guidance towards an approximate time-scale. In particular, reports of $5\%$ mortality after two years of infection \cite{kessler2012dieback}, $75\%$ mortality within five years \cite{langer2015ash} and no observations of infected ash surviving beyond $15$ years \cite{wylder2018evidence}. However, as discussed in detail later, the main results of this chapter depend on time-scales below the mean infection life-time. So once again, there is some degree of flexibility in the precise time-scale of $I_{n}\rightarrow R$.


% subdividing compartments in this manner could also provide an easier implementation to hosts which become more infectious, through a greater production of spores, as the infectious cycle continues not to mention particular periods of environmental unsuitability.
% 

 
\subsection{Dispersal parameterisation}

Dispersal was informed by data collected in France \cite{grosdidier2018tracking} and provide fitted dispersal parameters for ADB using Gaussian and inverse power-law dispersal. Notably, the authors studied dispersal at two different spatial scales, regional and local. For a review on the importance of scale, see chapter \ref{chapter2:litreview}. The experimental data analysed by \cite{grosdidier2018tracking} tracked ADB spores about known sources of infection. Although data on spore depositions does not necessarily correspond to a new infection, it sheds light on the spatial scale of ADB dispersal. 
\begin{figure}
    \centering
    \includegraphics[scale=0.25]{chapter6/figures/fig2.pdf}
    \caption{Dispersal gradients at the local spatial scale, parameters are informed by the work of \cite{grosdidier2018tracking}. The authors provide estimates for Gaussian and inverse power-law kernels: $\ell_{ga} = 195\mathrm{m}$ and ($\ell_{pl} = 205\mathrm{m}$, $a=3.3$) respectively, where $a$ represents the inverse power-law exponent.}
    \label{fig:dispersal-parameterisation}
\end{figure}

Given that the scale of interest in the $SEIR$ model is small, at a  resolution of $5\mathrm{m} \times 5\mathrm{m}$, a decision was made to parameterise dispersal with the local-level dispersal values given by \cite{grosdidier2018tracking}. See table \ref{tab:SEIR-model} and Figure \ref{fig:dispersal-parameterisation}. This decision supposes that the spread of ADB in Great Britain is comparative to the spread of ADB in France. Nonetheless, there is a noticeable difference in the French climate, landscape and, importantly, wind patterns. % references
Notwithstanding these differences, it stands to reason that dispersal parameters, if measured over a smaller, local spatial scale, would be less pronounced. 

Furthermore, the regional-scale dispersal analysed by \cite{grosdidier2018tracking} measures the spread of ADB over spatial scales of $10$-$100$'s of kilometres and is thought to contain artefacts of LDD by human-mediated transport. The regional spread of ADB though LDD is, therefore, beyond the scope of the present chapter\textemdash a discussion of LDD will conclude this chapter and lead us to chapter into chapter \ref{ch7:pde}.

\section{Informed ash density maps}

Ash densities are parameterised by abundance data provided by \cite{hill.data}. Previously, an oak canopy cover given by \cite{hill.data} was used in the toy model of chapter \ref{fig:ch4_uk_spread} and motivated the construction of a generic $SIR$-based dispersal model in chapter \ref{ch5:dispersal-model}. The canopy cover datasets produced by \cite{hill.data} combine data from different sources and regression methods are used to extrapolate canopy cover over the whole of Great Britain\textemdash for a more in-depth description of the methods used by \cite{hill.data} and a review of host data in general, see chapter \ref{chapter:lit-rev}.

Conveniently for us, ash happened to be among the most accurate data sets given by \cite{hill.data}, but two modifications had to be made to complement the $SEIR$ model. Initially, the data set had a resolution of $\mathrm{1km}\times \mathrm{1km}$ and units of hectares of canopy cover per kilometre-squared of land $\mathrm{ha/km^2}$. Firstly, the raw abundance values were re-scaled into a dimensionless tree density $\rho$. The same process was outlined in chapter \ref{ch5:dispersal-model} i.e. by converting the units $\mathrm{ha/km^2}$ to kilometre-squared of ash cover per kilometre-squared of land. 

Secondly, the resolution was re-scaled to $\mathrm{5km}\times \mathrm{5km}$, as shown in Figure \ref{fig:ash-host-data}(a). The purpose of re-scaling the map is a direct consequence of local wind-borne dispersal. In the local scale field study conducted by \cite{grosdidier2018tracking}, fungal spores were detected at a maximum distance of $500\mathrm{m}$. In a domain of resolution$\mathrm{1km}\times \mathrm{1km}$, it would therefore be tangible for spores to disperse straight over individual patches and infect non-nearest neighbours. However, increasing the spatial extent of individual land patches to $\mathrm{5km}\times \mathrm{5km}$ provides some level of assurance that local scale wind-dispersed spores cannot so simply jump over individual patches. There is always the possibility that spores travel larger distances, from mainland Europe to Great Britain for example \cite{freer2017tree, wylder2018evidence}. Yet, this chapter is predicated on targeting dispersal at local scales, and such detail can be omitted for now.

The frequency distribution is shown in Figure \ref{fig:ash-host-data}(b) reveal that the average density of ash in Great Britain is $\rho=0.017$. Furthermore, few locations support densities of $\rho=0.10$ and over\footnote{In the original $2.2\times 10^4$ $1\mathrm{km^2}$ data points, there were a handful of outlier points with densities in the interval $\rho \in [0.10, 0.30]$ which were excluded from the analysis.}.  Between the limits of $\rho \in [10^{-2}, 10^{-1}]$, the frequency distribution follows a power law of the form $\sim \rho ^{-k}$, as evident from the linearity on the logarithmic inset axes. The distribution had a fitted exponent of $k=1.90$, shown by the dashed black line. Intriguingly, this observation is suggestive of self-similarity in the data. Figure \ref{fig:ash-host-data}(a) shows that the south of England contains the highest concentration of high-density ash patches. Ash becomes progressively less abundant in Scotland and coastal locations in western Whales.

% \footnote{A review on the landscape epidemiology of ADB \cite{doi:10.1111/1365-2745.13383} concluded that the effect of host abundance in the neighborhood of ADB decayed with distance, according to an exponential distribution with scale parameter $200\mathrm{m}$. That is, the severity of ADB symptoms depended on the surrounding abundance of ash.} 
% 

\begin{figure}
    \centering
    \includegraphics[scale=0.30]{chapter6/figures/fig3-ash-data.pdf}
    \caption{Ash canopy cover data, as modelled by \cite{hill.data}. (a) the original resolution of $1\mathrm{km} \times 1\mathrm{km}$ was re-scaled to $5\mathrm{km} \times 5\mathrm{km}$ (b) the distribution of ash density over the re-scaled map of Great Britain, the inset shows a power-law behaviour.} % This figure is a place holder. (a) and (b) need to reflect the $5\mathrm{km} \times 5\mathrm{km}$
    \label{fig:ash-host-data}
\end{figure}


\section{Defining an $R_0$}

% R0 is the measure used to capture the pathogens ability to invade
% Q: what is the relationship between the mortality ratio (or final fraction of infected hosts) ? 
% If R0 was considered over a larger time-scale, host-regrowth would need to be integrated into the model
% Furthermore, measuring R0 over a larger time-scale could over, or underestimate, the degree to which a response would need to be undertaken in the response time-window. 
% a basic reproduction number $R_0$ is defined for the first life-cycle of the pathogen.
% Moreover, data over a relatively short time scale is typically all that is available when making decisions about control, not to mention the added complexity of incorporating host-regrowth–which becomes an important factor over longer time scales–and so, the constraint of computing R0 over longer time scales is relaxed. 
%  We develop the simplest implementation possible, namely, where R_0 is measured over one pathogen life-cycle, and the course of infection for each host follows the same process with perfect fidelity. This lies in stark contrast to a realistic scenario whereby different hosts show considerable differences.

\begin{itemize}
    \item see for a review on how R0 is calculated \cite{perspectives-on-r0}, we use method x in order to estimate and inform R0
    \item $R_0$ is a complex function which changes in time, to this end, the next generation operator is used to derive a value for $R_0$ \cite{doi:10.1098/rsif.2009.0386}. In order to inform the value of $R_0$ we do xy and z...
    \item Although it is hard to enforce a true $R_0$ value, the most important feature introduced from the definition is a threshold from which we can see if a local invasion is likely to take place.
    \item Since the number of susceptible hosts is fixed, without replacement, the number of susceptible hosts will continually decrease in time in the case of an epidemic. Given the strong spatial component in the model we expect that measuring $R_0$ as-per this definition will give an under-estimate for later times when few susceptible trees remain. Likewise, for earlier times when susceptible neighbours are plentiful $R_0$ will yield an upper estimate for the pathogen. In order to characterise the pathogen in this model, we take the upper bound defined between the $1^{st}-2^{nd}$ generations. This simplification represents a worst-case scenario and the mean value of $R_0$ would be lower for later times when there are less trees available to infect. But most importantly, the threshold of transmission $R_0>1$ is reliable captured \textit{during the initial stage of infection}.
    \item $R_0$ can be fully characterised by a growth rate \cite{R0-construct}. That is, the growth factor per generation.
    
\end{itemize}

\section{Seasonal $SEIR$ model behaviour}
% discuss stochastic model behaviour etc.
At its most basic level, we consider two sporulation functions and two dispersal kernels. 

\subsection{Sporulation}
% discuss multiple sporulation functions

% A simple time-dependant function $\phi(T)$ takes the value of $1$ between the summer months of June - September, and $0$ otherwise thus reflecting the sporulation peak of \textit{Hymenoscyphus fraxineus} (HF). During these summer months, wind-borne dispersal occurs when fruiting bodies produce large numbers of ascospores. 


\section{Constructing $R_0$-maps}
% discuss  projecting R0-maps
% Assumption, there is considerable genetic variation in susceptibility that varies between spatial location \cite{stocks2017first}
% \begin{figure}
%     \centering
%     \includegraphics[scale=0.25]{chapter6/figures/fig2.pdf}
%       \caption{(a) Ensemble-averaged $R_0$ between two values of $\beta$, for Gaussian dispersal of $\ell=1.6\mathrm{km}$. (a-b) $R_0$-map for $\beta=3\mathrm{e}-05$, coarse-grained at $5$, $ 10\mathrm{km^2}$. (c-d) $R_0$-map for $\beta=6\mathrm{e}-05$.}
%     \label{fig:my_label}
% \end{figure}

\section{Optimal fragmentation of $R_0$-maps}


\section{Chapter summary}

\textbf{Model improvements...}
\begin{itemize}
    \item 
    \item introduce the regime of pathogen spread we are interested in, we are not modelling continental long-range spread via upper atmosphere, nor are we interested in long-range human trade networks. We are interested in dispersal at the local level whereby passive means of wind, soil and or insects/animals. 
\end{itemize}


% Where we consider spatial structure and others have not
% - In this chapter, we focused on disrupting local-level dispersal by implementing control at landscape-level.
% -  In this chapter we undertook the ambitious task of scaling up a small-scale model, to large spatial distances covering GB, and developed a new approach to controlling the spread of disease.

These results are focused toward help policy makers implement informed decisions, about \textit{where} to control the spread of disease, based on spatial arguments, when budgets are low\textemdash as noted by \cite{time-varying-infectivity}. Our approach of spatially scaling small-scale epidemiological principles was conducted through computational, opposed to analytical, means. 

The article published by \cite{time-varying-infectivity} indicates the possibility of \textit{preferentially} controlling an area based on the final-sized epidemic. It goes without saying that areas of land that have the largest final sized epidemic are likely the most density populated. However, we outline the possibility it might be more beneficial to preferentially control an area based on its spatial location and how it couples to to neighbouring areas.

% The scaling up of our model
The scaling up of our model resembles a metapopulation model, now commonplace when modelling plant-based epidemics, but crucially our large-scale model is not dynamic. A dynamic large-scale model is indeed useful for the prediction of time-scales and the movement of disease-fronts movement; however, we suggest they might be at best less useful for optimised large-scale control, and at worse, nicely complemented by the approach we develop.

If early data through a particular region, with known data, was collected, it would be a relatively short step to fit the dispersal model and scale up the model over large distances given the seemingly simple relation to host density $\rho$.
% At the small scale we have a uniform population. However for larger scales we have considerable spatial heterogeneity.
% spatial-scale and control, over a multi-seasonal pathosystem, have been investigated for sugar beet in the UK, see \cite{doi:10.1146/annurev.phyto.45.062806.094357} for a review.

% On the SEIR model

The cyclic nature of the $SEIR$-type model constructed in this article can also find resemblance in the, well established body of literature, of crop-based epidemics. It is common-place for a field of crops to become infested, then at the end of the growing season be totally eradicated by virtue of harvesting % \cite{time-varying-infectivity} has 
% we contend the SnIEmR model, considered over one sporulation peak, is a simplistic implementation of the SEIR-based model needed to compute an invasion threshold that represent the infection dynamics ash dieback. 
% The SEIR-based model is made more flexible, and can be readily extended or adapted to incorporate more biological realism, by virtue of splitting the E and I into multiple compartments\textemdash this is frequently done in human and animal-based models \citations..

%- Although the main result of our work was conduced over one sporulation peak, or life-cycle of ash dieback, splitting the model into various compartments was a useful and necessary step towards developing accurate large-tree species. reference  \cite{https://doi.org/10.1111/ppa.12894} alludes to multiple exposed periods being useful

%- A quantity of interest that appears frequently in the literature is the initial growth rate $r$, that is the density-independent growth at the start of any epidemic.
% See \cite{ferrandino2012time} for a review on the time-scales and sporulation of plant-based diseases.

% Fitting data to the early phases of an epidemic has been shown to give large differences in the final-size epidemic, or severity \cite{time-varying-infectivity}. However, we are only interested in a pathogens ability to invade and this can be closely approximated by measuring R0 over the fist observation (\see appendix)

% The variability between the particular life-cycle history followed by hosts is thought to be an important factor to consider\cite{ferrandino2012time}, and we intentionally neglected this for the purpose of simplicity...

% Accurately modelling time-varying infectivity is difficult \cite{13-challenges}, and as such, we opted for parsimony.

% Main assumptions in the method: 
% 1) Assumptions about dispersal
Dispersal over small spatial-scale is thought to predominately occur passively through wind, however other means of dispersal exist such as soil and or insects. There are many pathways a pathogen can use to spread through a landscape, including long-range dispersal, mediated through either human-trade networks %
\cite{hulme2009trade, banks2015role, chapman2017global} or dispersal in the upper atmosphere \cite{westbrook1999atmospheric, isard2005principles}.  

% We assumed some regions can sustain an epidemic and some cannot, we assume that $R_0>1$ is the threshold separating these regimes.

% 2) Assumptions about R0
To our knowledge, $R_0$ has not been estimated for Ash dieback, or indeed for any large deciduous tree species, unlike for some crop-based pathogens \cite{segarra2001epidemic}. The lack of $R_0$-estimates made it hard to scrutinize which $\beta$-valued $R_0$-map would be likely to reflect reality. This gap in the body of literature is hardly surprising given the complexity of measuring time-varying infectivity rates \cite{13-challenges}. Thus, as it stands, our results hint-towards the utility of landscape-level control but come short of definitive proof.

% Looking at \cite{R0-perc-ref}, it makes me think our notion of $R_0$ is pretty simplistic. We only measure the local-level $R_0$. We do not consider $R_0$ from patch to patch. What scale we measure $R_0$ has a huge impact on what the result is. Could we rank land-patches not only on there local $R_0$ level, but also on the impact they have on there immediate neighbours ? This would, in theory be an improvement to the clustering algorithm.

% Improvements to the model:
% 1) The algorithm
% The algorithm to target not only the critically connecting patches, but also find fragmenting lines which minimise risk at the landscape-level ? Incorporating the local impact a particular patch may have on its neighbours.
% 2) Multi-year R0 analysis
% However, we .... xyz cover all basis of using a one-season approximation. <- this leads to a risk-based argument in which we could capture a 2nd-order R0 which does, xyz. We mainly interested in a pathogens ability to invade. 

% Analysis was aired towards simplicity, a more expansive study with e.g. more sophisticated sporulation functions could be the subject of future work.

% The most important message of our work was...
% The spatial and temporal scale of the control-strategy should match intrinsic spatial and temporal scale of the invasion. <- Hence R0 measured over one season. 


% our modelling approach and results are in their infancy,  

% Last remarks and future modelling work:
To our knowledge, there is a surprising lack of spatio-temporal ADB models exist in the literature, probably because of the significant challenges involved in containment. Although our findings are far from complete, it suggests the spread of ash dieback, between spatial locations, could be reduced by preferentially targeting sites to minimise epidemiological connectivity. Not surprisingly, more work will need to be done to ascertain the degree to which this strategy could impede the spread. 

% Crucially, future work will involve integrating LDD mechanisms into the model in order to understand the relative importance long vs local distance dispersal. We may speculate about the relative importance looking at figure x, whereby the maximum distance spread in season due to local-scale spread is xm/year, in stark contrast from the observed spread of 40-60km/yr.

% We cannot overstate the importance of LDD, and it is hard to say the degree to which targeting the local dispersal mechanism alone will inhibit the spread. We will revisit this question in future work, however, we contend that preferentially targeting diseased trees based on spatial location.....could help control epidemics with greater efficacy. 

% We may speculate how our result could aid the effort of choosing where to re-plant ash stands genetically engineered to be less susceptible; if re-planting efforts were undertaken in certain location.... <- speculative

% We may speculate about how persistent ash dieback would be, even if a large-scale control effort was undertaken

% There is evidence to suggest regional variation in mortality due to ash dieback \cite{stocks2017first}, this could be incorporated into the model...

% Our results support the call for more research to be undertaken into multi-scale dispersal, 

% Recently, it has been suggested that the dispersal-kernel of wind-borne pathogens might follow a scaling law \cite{https://doi.org/10.1111/jbi.13642}. The significance of such a finding would allow us to analyse the $R_0$-maps over much more flexible spatial scales. % Explain.

% This sentence is wrong, the cited paper makes an argument for the spatially-scaling up of dispersal kernels, which happens to still be useful paper to cite, re-phrase and re-frame accordingly.

% see \cite{ash-dieback-costs}, and the references therein (methodology excel spread sheet S1), for mortality references the latest findings suggest a mean mortality rate of 95%.



