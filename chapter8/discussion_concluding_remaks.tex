
\chapter{Discussion}

In this thesis, a simple model of forest epidemics was incrementally extended into a more elaborate framework.
Each incremental improvement led to a novel, general-purpose framework to visualise epidemic severity across GB.
The framework is computationally efficient and adaptable to any wind-dispersed tree pathogen, provided a sufficient host density distribution is available. Conclusively, more research is required to progress and test the model against observational disease incidence data.
However, several exciting research avenues emerge from the work conducted in this thesis\textemdash discussed more below.

After setting the scene with a simple lattice model (SLM) of tree disease in Chapter \ref{chapter:SLM}, 
Chapter \ref{chapter:SLM-applications} linked the SLM with a map of predicted oak abundance given by \cite{hill.data} to produce a large-scale `toy' model. 
A key result emerged from Chapter \ref{chapter:SLM-applications}.
Namely, that \acrshort{nn} interactions were fundamentally insufficient to describe the spread of disease across lower, more realistic landscape tree densities. 
Fortunately, Chapter \ref{ch5:dispersal-model} resolved the issue by constructing a non-local dispersal model.
Nonetheless, Chapter \ref{ch5:dispersal-model} also demonstrated that a small dispersal length scale can still gives rise to an unnatural, wave/percolation-like epidemic. 
At the very least, these results bolster the growing body of research highlighting the importance of dispersal and help guide modellers to construct more representative models in botanical epidemiology.

Additionally, Chapter \ref{ch5:dispersal-model} examined two methods of calculating $R_0$ for a non-local dispersal model (NLM) over a range of epidemic parameters. Results from Chapter \ref{ch5:dispersal-model} indicate that when the scale of dispersal is comparable to the domain, the model approximates a mass-action well-mixed system described by the standard SIR model. However, comparisons to the SIR model were simplified and limited to one parameter (i.e. the ratio $\beta/\gamma$). Subsequently, the analysis constitutes a preliminary result, and a more sophisticated comparison method is required to glean further insight.

Most intriguingly, the spatially-explicit derivation of $R_0$ led to an `entire function' in Equation \ref{eq:ein}, a well-known function in complex analysis \cite{abramowitz1948handbook}. Entire functions have been examined in several theoretical settings (e.g. \cite{littmann2005entire, hryniv2009zeros, sixsmith2011entire}), including a spatially-structured population model with dispersal \cite{zhang2017non}. However, despite a detailed search, no theoretical studies expand or develop upon Equation \ref{eq:ein} in the context of epidemics. 
As such, Equation \ref{eq:ein} marks a different approach for determining $R_0$ that considers the spatial structure, host density, dispersal mechanism and infectivity. Further theoretical studies could explore a more rigorous analysis of Equation \ref{eq:ein} and perhaps examine an alternate derivation incorporating inverse power law dispersal with exponentially-distributed removal lifetime dynamics. However, the mathematical derivation might become too challenging in the face of more complicated epidemic models

Then in Chapter \ref{ch5:dispersal-model}, the analytic expression of $R_0$ was compared against the `actual' contact-traced reproduction ratio. Both methods used to determine the reproduction ratio demonstrated a clear epidemic threshold at $R_0=1$, marking an important finding. Nevertheless, comparing both methods revealed that the analytic expression overestimates $R_0$ for progressively higher values ($R_0 \sim 10$), highlighting a significant limitation in the approach. 
Had the analytic expression been applied to a dense forest or highly aggregated distribution (where $\rho >10\%$ and $R_0$ is likely higher), the overestimation would only increase.
The utility of Equation \ref{eq:ein} therefore diminishes when used to describe highly infectious regimes but serves as an accurate approximation around the threshold $R_0 \approx 1$. Notwithstanding,
tree-based diseases are unlikely to reach such exceptionally high epidemic values (where $R_0>10$), as supported by estimates of $1 \lessapprox R_0 \lessapprox 6$ for the Dutch elm disease epidemic in GB \cite{swinton1996dutch} and $1 \lessapprox R_0 \lessapprox 4$ for oak processionary moth epidemics in London \cite{wadkin2022inference}.

In contrast, the contact-tracing method computes the mean number of infections per infected tree over different infected generations.
Computing the mean-generational reproductive ratio proved convenient and led to a sharper epidemic threshold for later generations. These observations relate nicely to \cite{R0-perc-ref}, who studied a similar method to calculate $R_0$ for foot and mouth disease.
Despite the convenience of contact-tracing secondary infections in the model, it would be challenging to measure in natural systems experimentally.
As such, the method remains applicable to abstract modelling work alone and not in-the-field experiments.

The central value of Chapter \ref{ch:6-adb} results from outlining a novel framework to link a wind-dispersed epidemic model with species abundance data. 
The framework focused on the fungus ADB, which is well-established and already spread throughout Europe and the UK.
Thus, unfortunately, attempting eradication at this stage of pathogen development is untenable \cite{ash-dieback-costs}. 
Still, Chapter \ref{ch:6-adb} marks a step towards a general framework that can help policymakers reach informed decisions about \textit{where} to focus epidemic control.
In particular, the framework could provide value as an approach to threat assessment and rapid response modelling during the early phase of an epidemic.

In addition, Chapter \ref{ch:6-adb} developed a simplified spatially-explicit SEIR model that described the seasonal spread of ash dieback (ADB) over local spatial scales. Subsequently, we coupled the ADB model to a map of predicted ash abundance given by \cite{hill.data} to produce an $R_0$-map that covered GB.
Under the threat of ADB, the UK government commissioned some early 'rapid response` modelling work as part of the Chalara management plan\footnote{As set by the government, readers can find more information by reading the Chalara Management Plan: https://www.gov.uk/government/publications/chalara-management-plan} \cite{defra2013chalara}. Although, the modelling work undertaken by  Chalara Cambridge plant sciences remains unpublished. Regardless, no compartmentalised model of ADB could be found in the literature. Therefore, to my knowledge, the ADB model presented in this thesis is the first mechanistic (compartmentalised) attempt to model ADB's epidemic spread.

Treating infectivity as a free parameter in Chapter \ref{ch:6-adb} led to several notable observations. When infectivity is low, the $R_0$-maps become sparsely populated with susceptible patches above the epidemic threshold. Consequently, the $R_0$-maps indicate that ADB would be unlikely to invade GB below a hypothetical minimum infectivity value.
In a similar vein, clusters in the $R_0$-map grew rapidly over four orders of magnitude between a narrow range of infectivity parameters.
Altogether, the model alludes to behaviour akin to a global epidemic phase transition across GB, though numerous assumptions place limitations on the framework.

As mentioned previously, infectivity ($\beta$) was unfitted to data and kept as a free parameter, enabling $R_0$-map analysis over a spectrum of infectivity parameters. 
However, arbitrarily defining infectivity underpinned a central limitation in work, and future research should aim to address this. 
Several works have published spatio-temporal and ADB mortality data in Europe \cite{https://doi.org/10.1111/1365-2745.13383, https://doi.org/10.1002/ppp3.11, stocks2017first, lohmus2014ash}, which could be used to estimate epidemic speed and lifecycle parameters using Bayesian Markov-chain-Monte-Carlo methods.
Similar approaches have been adopted to infer the spread of SOD in California \cite{10.1371/journal.pcbi.1002328} and citrus canker in Florida \cite{neri2014bayesian}, though no such work has been undertaken for ADB.
Further research is ultimately required to confirm the suitability of statistically fitting data to the spread of ADB, yet the prospect remains positive given the quantity of available data.

ADB mortality rates depend heavily on the landscape, composed of either rural, woodland or urban settings.
The type of environment is one particular consideration relevant for future epidemic parameter inference studies.
For simplicity, Chapter \ref{ch:6-adb} neglected environment types and employed a one-to-one mapping between infectivity and each $R_0$-map. In a more sophisticated model, each gird in the $R_0$-map would depend on a $\beta$ parameter dependent on the type of environment.
Thus, statistical inference based on the environment type goes hand-in-hand with improving accuracy in the $R_0$-map.
The national forest inventory could conveniently aid these future improvements, as it holds relevant information on which regions are woodland, forest or urban\textemdash discussed previously in section \ref{sec:nationa-surveyes}.

Chapter \ref{ch7:landscape-level-control} outlined the first steps toward epidemic control predicated on the host spatial structure. That is, identifying and targeting positions in the host distribution that may disrupt epidemic dispersal between regions. Initial results reveal that epidemic connectivity can depend on a small number of `connecting' positions in the host distribution. Nevertheless, more research is required to assess the utility and efficiency of the control strategy. Chapter \ref{ch7:landscape-level-control}
outlined a potential research direction to examine and test the control strategy by considering a set of coupled patches in section \ref{sec:future-questions}. From the coupled system, the transmission probability and effect of control between host patches can be assessed. Until this work is undertaken, the strategy remains speculative.
Following the research direction posed in section \ref{sec:future-questions}, future work should develop the definition of connectivity inside the $R_0$-map. Each pixel within an $R_0$-map reflects only isolated within-patch interactions and not between-patch LDD. As such, there is no non-local connectivity between pixels in the map, 
a limitation clearly revealed when analysing clusters at different landscape resolutions in section \ref{sec:gaussian-r0-clustering}.

Unsurprisingly, insufficient host data underpins a significant limitation in this thesis.
Although the predicted ash abundance map captured the overarching large-scale distribution of ash in GB, Hill et al. reported a RMSE of $5\mathrm{ha}$. Therefore, in reality, regions below the threshold might be susceptible\textemdash and vice-versa for above threshold regions. Unfortunately, such errors mean that the host distribution is not accurate enough to inform the hypothetical (fine-scale) management scenarios presented in Chapter 7. Until species abundance data captures the host distribution more reliably, country-wide applications remain extraordinarily ambitious. In response to this limitation, future research could concentrate on smaller-scale areas where abundance data is known. For example, by examining well-surveyed areas inside the UKCEH Countryside Survey data.

The regional containment strategy of Chapter 7 could help enhance the government's contingency and preparedness planning for new invasions. Presently, DEFRA lists contingency measures for numerous botanical diseases on the plant health portal\footnote{DEFRA's current set of contingency plans can be found at the following address: https://planthealthportal.defra.gov.uk/pests-and-diseases/contingency-planning/.}. Generally, contingency measures describe `demarcated zones', consisting of an `infected area' and a `buffer zone'; this is confirmed by reading DEFRA's contingency plans for \textit{Xylella fastidiosa}, oak wilt, oak processionary moth and emerald ash borer. Infected areas outline  ($r \lessapprox 100 \mathrm{m}$) regions directly surrounding verified infections where the destruction of susceptible plant material is recommended. Subsequently, a buffer zone ($1 \mathrm{km} \lessapprox  r \lessapprox 2.5 \mathrm{km}$) is established around the infected zone. Buffer zones are then subject to continuous surveillance and monitoring to detect new infections. However, no initiative exists to coordinate epidemic control between different infected zones. Instead, contingency plans aim to manage infected/buffer zones independently, which contrasts with the regional control strategy proposed in Chapter 7.

In a large-scale outbreak with several confirmed infected areas, the method illustrated in Chapter 7 could help prioritise which infected sites undergo epidemic control, with the added benefit of effectively slowing the spread between regions. For example, regional containment would be advantageous when an emergent infectious epidemic threatens to invade an uninfected yet high-risk neighbouring area. Another use-case pertains to a scenario where outbreaks established close to the coastline threaten high-risk regions situated more inland, as illustrated by Figure \ref{fig:payoff-efficiency}(e). In any case, slowing the spread between regions gives tree populations added time to recover and offers policymakers and stakeholders vital time to respond. 

In conclusion, the research narrative developed in this thesis aims to help inform policymakers about where to focus epidemic control.
The approach constitutes an epidemic mapping framework for tree disease with parallels to the emerging field of Infectious Disease Cartography in human epidemiology. 
The framework is computationally efficient, flexible, and adaptable to other pathosystems.
Several theoretical insights were ascertained from deriving a spatially-explicit expression for $R_0$ and comparing it against a stochastic non-local dispersal model. Lastly, a novel epidemic control strategy was outlined, though more work is needed to progress the framework and rigorously validate results.