
\chapter{Discussion}

In this thesis, a simple model of forest epidemics was incrementally extended into a more elaborate framework.
The incremental improvements led to a novel, general-purpose framework to visualise epidemic severity across GB.
The framework is computationally efficient and adaptable to any wind-dispersed tree pathogen, provided a sufficient host density distribution is available. Conclusively, more research is required to progress and test the model against observational disease incidence data.
However, several exciting research avenues emerge from the work conducted in this thesis\textemdash discussed more below.

After setting the scene with a simple lattice model (SLM) of tree disease in Chapter \ref{chapter:SLM}, 
Chapter \ref{chapter:SLM-applications} linked the SLM with a map of predicted oak abundance given by \cite{hill.data} to produce a large-scale `toy' model. Consequently, a key result from Chapter \ref{chapter:SLM-applications} demonstrates that nearest-neighbour interactions are
fundamentally insufficient to describe the spread of disease across lower, more realistic landscape tree densities. 
Furthermore, even with non-local dispersal, Chapter \ref{ch5:dispersal-model} demonstrates that an epidemic propagating with a small dispersal length scale gives rise to an unnatural, highly wave-like pattern. Together, these results guide modellers to construct more representative models and bolster the growing body of research that highlights the importance of dispersal in botanical epidemiology.

Chapter \ref{ch5:dispersal-model} examined two methods of calculating $R_0$ for a non-local dispersal model (NLM) over a range of epidemic parameters. Results from Chapter \ref{ch5:dispersal-model} indicate that when the scale of dispersal is comparable to the domain, the model approximates a mass-action well-mixed system described by the standard $SIR$ model. However, comparisons to the $SIR$ model were simplified and limited to one parameter (i.e. the ratio $\beta/\gamma$). Subsequently, the analysis constitutes a preliminary result, and a more sophisticated comparison method is required to glean further insight.

Most intriguingly, deriving a spatially-explicit analytic expression of $R_0$ led to an `entire function' in Equation \ref{eq:ein}, well-known in complex analysis \cite{abramowitz1948handbook}. 
Equation \ref{eq:ein} contrasts with the majority of approaches to calculate $R_0$ that generally rely on infectivity/removal rates without including spatial structure or dispersal. Instead, Equation \ref{eq:ein} is a function of infectivity, removal, dispersal, and tree density parameters. Further theoretical studies could explore a more rigorous analysis of Equation \ref{eq:ein}, or examine an alternate derivation incorporating inverse power law dispersal or exponentially-distributed removal lifetime dynamics.
However, it is clear to see that the mathematical derivation might ultimately become too challenging in the face of more complicated epidemic models. 

Then in Chapter \ref{ch5:dispersal-model}, the analytic expression of $R_0$ was compared against the `actual' contact-traced reproduction ratio. Both methods to determine the reproduction ratio demonstrated a clear epidemic threshold at $R_0=1$, thereby marking an important finding. Nonetheless, comparing both methods revealed that the analytic expression overestimates $R_0$ for progressively higher values ($R_0 \sim 10$), highlighting an important limitation in the approach. 
Had the analytic expression been applied to a dense forest ($\rho >10\%$) where $R_0$ is likely higher, overestimation would only increase.
Therefore, the value of Equation \ref{eq:ein} rapidly diminishes when used to describe highly infectious regimes with a high $R_0$.

In contrast, the contact-tracing method computes the mean number of infections per infected tree over different infected generations.
Computing the mean-generational reproductive ratio proved convenient and led to a sharper epidemic threshold for later generations. These observations relate nicely to \cite{R0-perc-ref}, who studied a similar method to calculate $R_0$ for foot and mouth disease.
Despite the convenience of contact-tracing secondary infections in the model, it would be challenging to measure in natural systems experimentally.
As such, the method remains applicable to abstract modelling work alone and not in-the-field experiments.

The central value of Chapter \ref{ch:6-adb} results from outlining a novel framework to link a wind-dispersed epidemic model with species abundance data. Although the framework focused on ADB, the pathogen has long since become well-established and spread throughout Europe and the UK, meaning that attempts at epidemic control are currently untenable. However, Chapter \ref{ch:6-adb} marks a significant step towards a general framework to help policymakers make informed decisions about \textit{where} to focus epidemic control.
Above all, the framework could provide value as an approach to threat assessment and rapid response modelling during the early phase of an epidemic.

In particular, Chapter \ref{ch:6-adb} developed a simplified spatially-explicit $SEIR$ model that described the seasonal spread of ash dieback (ADB) over local spatial scales.
Subsequently, the ADB model was coupled with the map of predicted ash abundance given by \cite{hill.data} to produce an $R_0$-map covering GB.  
Surprisingly, no compartmentalised model of ADB could be found in the literature. Thus, to my knowledge, the ADB model presented in this thesis 
is the first mechanistic attempt at modelling the epidemic spread of ADB.

Treating infectivity as a free parameter in Chapter \ref{ch:6-adb} led to several notable observations. When infectivity is low, the $R_0$-maps become sparsely populated with susceptible patches above the epidemic threshold. Consequently, the $R_0$-maps indicate that ADB would be unlikely to invade GB below a hypothetical minimum infectivity value.
In a similar vein, between a narrow range of infectivity parameters, clusters in the $R_0$-map grew rapidly over four orders of magnitude.
Altogether, the model alludes to behaviour akin to a global epidemic phase transition across GB. Unfortunately, however, numerous assumptions place limitations on the framework.

As mentioned previously, infectivity ($\beta$) was unfitted to data and kept as a free parameter, thereby permitting the $R_0$-map analysis over a spectrum of infectivity parameters. However, arbitrarily defining infectivity posed a significant limitation in the framework, and future research should aim to address this. In addition, several publications have recorded spatio-temporal ADB mortality data in Europe \cite{https://doi.org/10.1111/1365-2745.13383, https://doi.org/10.1002/ppp3.11, stocks2017first, lohmus2014ash} that could form the basis for Bayesian Markov-chain-Monte-Carlo inference studies to estimate epidemic speed and lifecycle parameters.
Although similar approaches have been adopted to infer the spread of SOD in California \cite{10.1371/journal.pcbi.1002328} and citrus canker in Florida \cite{neri2014bayesian}, no such work has been undertaken for ADB.
Hence, further research is ultimately required to confirm the suitability of statistically fitting data to the spread of ADB. 

ADB mortality rates depend heavily on the landscape, composed of either rural, woodland or urban settings.
The type of environment is one particular consideration relevant for future research on epidemic parameter inference.
For simplicity, Chapter \ref{ch:6-adb} neglected environment types and employed a one-to-one mapping between infectivity and each $R_0$-map. In a more sophisticated model, each gird in the $R_0$-map would depend on a $\beta$ parameter dependent on the type of environment.
Thus, statistical inference based on the environment type goes hand-in-hand with improving accuracy in the $R_0$-map.
Furthermore, the national forest inventory could conveniently aid any future improvements, as it holds relevant data on which regions are woodland, or forests\textemdash
discussed previously in section \ref{sec:nationa-surveyes}.

Chapter \ref{ch7:landscape-level-control} outlined the first steps toward epidemic control predicated on the host spatial structure.
That is, identifying and targeting positions in the host distribution that may disrupt epidemic dispersal between regions.
Initial results reveal that epidemic connectivity can depend on a small number of `connecting' positions in the host distribution.
Nevertheless, more research is required to assess the utility and efficiency of the control strategy. Chapter \ref{ch7:landscape-level-control}
outlined a potential research direction to examine and test the control strategy by considering a set of coupled patches in section \ref{sec:future-questions}.
From the coupled system,  the transmission probability and effect of control between host patches can be assessed.
Until this work is undertaken, the strategy remains speculative.

Following the research direction posed in section \ref{sec:future-questions}, future work should develop the definition of connectivity inside the $R_0$-map. Each pixel within an $R_0$-map reflects only isolated within-patch interactions and not between-patch LDD.
As such, there is no non-local connectivity between pixels within each $R_0$-map. Section \ref{sec:gaussian-r0-clustering} revealed the same limitation by analysing clusters at different landscape resolutions.

Unsurprisingly, insufficient host data underpins a significant limitation in this thesis.
Although the predicted ash abundance map captured the overarching large-scale distribution of ash in GB, Hill et al. reported a RMSE of $5\mathrm{ha}$. Therefore, in reality, regions below the threshold might be susceptible\textemdash and vice-versa for above threshold regions.
Unfortunately, such errors mean that the host distribution is not accurate enough to inform the hypothetical (fine-scale) management scenarios presented in Chapter 7. Until species abundance data captures the host distribution more reliably, country-wide applications remain extraordinarily ambitious. In response to this limitation, future research could concentrate on smaller-scale areas where abundance data is known. For example, by examining well-surveyed areas inside the UKCEH Countryside Survey data\textemdash reviewed in section \ref{sec:nationa-surveyes}.

In conclusion, the research narrative developed in this thesis aims to help inform policymakers about where to focus epidemic control.
The approach constitutes an epidemic mapping framework for tree disease with parallels to the emerging field of Infectious Disease Cartography in human epidemiology. Furthermore, the framework is computationally efficient, flexible, and adaptable to other pathosystems.
Several theoretical insights were ascertained from deriving a spatially-explicit expression for $R_0$ and comparing it against a stochastic non-local dispersal model. Lastly, a novel epidemic control strategy was outlined, though more work is needed to progress the framework and rigorously validate results.

% The framework produced epidemic maps  differs from other large-scale approaches to model the spread of tree disease 
% Infectious Disease Cartography, one seeks to map the likelihood, or risk, of infectious disease outbreaks and produce risk-maps.
% Infectious Disease Cartography
% Disease cartography using 

% \begin{itemize
%     \item We have...Disease cartography using liking species abundance with spatially-explicit epidemic model. A notable challenge posed by a dispersal-based system
%     is that regions of land couple together non-locally. Throughout this thesis, CCA assessed clustering based on neighbour links. 
%     Going forward, a more sophisticated notion of connectivity could permit higher resolution maps for Inverse power law... method could add significant value... 
%     \item focus on a small study area, as in \cite{he2019integrating}
% \end{itemize}

%---------------------------------------------------------
% relation of work to more commonplace Metapopulation models
% In the process of computing a value for $R_0$, several insights into dispersal-based epidemic progression and spatial-scale came to light.
% Namely, computing $R_0$ for a fat-tailed inverse power law dispersal kernel required a larger to domain-size in comparison to the thin-tailed Gaussian variants.
% A larger domain-size necessitated a lower landscape-level resolution (or equivalently larger patch-size) when forming the $R_0$-map.
% Thus to define an $R_0$-value in the framework, landscape-level patch-size needs to reflect the nature of dispersal;
% metapopulation-type models do not have this limitation. 
% Yet, at the same time typical metapopulation models are not predicated on small-scale epidemiological properties, 
% but tend to model patch-sizes on the order of $100\mathrm{m}-1\mathrm{km}$ \cite{large-scale-control, doi:10.1111/j.1365-3059.2010.02391.x}.

%---------------------------------------------------------
% generalisation to the approach
% In favour of parsimony, infectivity $\beta$ was kept constant over the entire landscape of Great Britain.
% However, epidemic severity of ADB is known to vary in response to either urban or rural environments \cite{marciulyniene2017can}.
% So in reality, $\beta$ may very well exhibit a spatial dependence.
% Likewise, including an index on, $\beta_i$, could reflect yearly changes due to climate change or habitat suitability\textemdash thus generalising the yearly infection cycles according to Figure \ref{fig:SEIR-transitions}.
% An attractive feature of the $SEIR$ model therefore includes the scope of its adaptability 
% and generalisations to our notion of $\beta$ could support more in-depth studies on spatio-temporal epidemic heterogeneity.

%-------------------------------------------------------
% - Crucially, future work will involve integrating LDD mechanisms into the model in order to understand the relative importance long vs local distance dispersal. We may speculate about the relative importance looking at figure x, whereby the maximum distance spread in season due to local-scale spread is xm/year, in stark contrast from the observed spread of 40-60km/yr.
% - We cannot overstate the importance of LDD, and it is hard to say the degree to which targeting the local dispersal mechanism alone will inhibit the spread. We will revisit this question in future work, however, we contend that preferentially targeting diseased trees based on spatial location.....could help control epidemics with greater efficacy. 
% - We may speculate how our result could aid the effort of choosing where to re-plant ash stands genetically engineered to be less susceptible; if re-planting efforts were undertaken in certain location.... <- speculative
% - We may speculate about how persistent ash dieback would be, even if a large-scale control effort was undertaken
% - There is evidence to suggest regional variation in mortality due to ash dieback \cite{stocks2017first}, this could be incorporated into the model...
% - Recently, it has been suggested that the dispersal-kernel of wind-borne pathogens might follow a scaling law \cite{https://doi.org/10.1111/jbi.13642}
