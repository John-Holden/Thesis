
\chapter{Discussion}
Throughout this thesis, we focused on the development of generalisable and flexible models that could, with accurate parameter-values, be tuned to model a wide variety of pests and pathogens. This thesis is not concerned with modelling any one type of tree disease, and only towards the end did we develop biological specificity towards ADB. Therefore, an in-depth biological review of a specific organism(s) was beyond the scope of this thesis.


The scope of this thesis is vast, and subsequently relatively general, although I presented several key themes.


% Interestingly, the disparity between clusters-size, and the distance between clusters could be used to give insight into
% which spatial scale is the dominant driver of disease spread.
% The more sparsely distributed\textemdash and thus fragmented\textemdash the $R_0$-map, the more important LDD becomes as the driver of between-cluster spread.
% Crucially, accessing the relative importance of spatial scale between disease drivers would rely on a fitted value of $\beta$,
% and the knowledge that ADB is estimated to spread with impunity and kill over $85\%$ of ash in Great Britain.


%---------------------------------------------------------
% comment on lack of ADB modelling
% To demonstrate the method, four simplified compartmentalised $SEIR$ models of ADB were constructed, composed of different sporulation and dispersal functions.
% Given the threat ADB poses to European ash, and the resulting economic impact, it was surprising to find a lack of spatio-temporal modelling work in the literature.
% A lack of dynamic spatial-explicit ADB models most likely reflect its complex life-cycle and lack of epidemic parameters.
% However, only in recent years has the sexual reproductive mode become widely accepted as the dominant disease-driver \cite{https://doi.org/10.1111/ppa.12844, havnavckova2017direct}.
% In addition to the life-cycle of ADB, multi-scale parameters for ascospore dispersal have only been published within the last few years \cite{grosdidier2018tracking}.

% To define invasibility over the landscape, we relied on a spatially-explicit value of $R_0$\textemdash i.e the mean number of infections produced by the first generation of infected ash.

%---------------------------------------------------------
% relation of work to more commonplace Metapopulation models
% In the process of computing a value for $R_0$, several insights into dispersal-based epidemic progression and spatial-scale came to light.
% Namely, computing $R_0$ for a fat-tailed inverse power law dispersal kernel required a larger to domain-size in comparison to the thin-tailed Gaussian variants.
% A larger domain-size necessitated a lower landscape-level resolution (or equivalently larger patch-size) when forming the $R_0$-map.
% Thus to define an $R_0$-value in the framework, landscape-level patch-size needs to reflect the nature of dispersal;
% metapopulation-type models do not have this limitation. 
% Yet, at the same time typical metapopulation models are not predicated on small-scale epidemiological properties, 
% but tend to model patch-sizes on the order of $100\mathrm{m}-1\mathrm{km}$ \cite{large-scale-control, doi:10.1111/j.1365-3059.2010.02391.x}.

%---------------------------------------------------------
% generalisation to the approach
% In favour of parsimony, infectivity $\beta$ was kept constant over the entire landscape of Great Britain.
% However, epidemic severity of ADB is known to vary in response to either urban or rural environments \cite{marciulyniene2017can}.
% So in reality, $\beta$ may very well exhibit a spatial dependence.
% Likewise, including an index on, $\beta_i$, could reflect yearly changes due to climate change or habitat suitability\textemdash thus generalising the yearly infection cycles according to Figure \ref{fig:SEIR-transitions}.
% An attractive feature of the $SEIR$ model therefore includes the scope of its adaptability 
% and generalisations to our notion of $\beta$ could support more in-depth studies on spatio-temporal epidemic heterogeneity.

%-------------------------------------------------------
% - Crucially, future work will involve integrating LDD mechanisms into the model in order to understand the relative importance long vs local distance dispersal. We may speculate about the relative importance looking at figure x, whereby the maximum distance spread in season due to local-scale spread is xm/year, in stark contrast from the observed spread of 40-60km/yr.
% - We cannot overstate the importance of LDD, and it is hard to say the degree to which targeting the local dispersal mechanism alone will inhibit the spread. We will revisit this question in future work, however, we contend that preferentially targeting diseased trees based on spatial location.....could help control epidemics with greater efficacy. 
% - We may speculate how our result could aid the effort of choosing where to re-plant ash stands genetically engineered to be less susceptible; if re-planting efforts were undertaken in certain location.... <- speculative
% - We may speculate about how persistent ash dieback would be, even if a large-scale control effort was undertaken
% - There is evidence to suggest regional variation in mortality due to ash dieback \cite{stocks2017first}, this could be incorporated into the model...
% - Recently, it has been suggested that the dispersal-kernel of wind-borne pathogens might follow a scaling law \cite{https://doi.org/10.1111/jbi.13642}

\begin{itemize}
    \item Data: modelled abundance data is not accurate enough to make such detailed, fine-scale management decisions, as presented in Chapter 7. According to \cite{hill.data},
    the RMSE for ash abundance distributions was $5\mathrm{ha}$. Given that connectivity, as we've defined it, can depend on a small number of points,
    large-inaccuracies could be expected in comparison to the identified areas presented in Chapter 7.
    \item Until modelled abundance data can identify species such as ash more accurately, future work should concentrate on modelling smaller-scale areas, 
    where abundance data is known. Country-wide applications are questionable, and extraordinarily ambitious at this point in time. However, several fronts could drastically improve the applicability of the framework presented in this thesis...
    \itme Although landscape management predictions to the individual patch is unrealistic, abundance maps are still expected to capture overall spatial distribution patterns,
    thus, $R-0$-maps can be expected to present the overarching pattern of pathogen invasiveness.
    \item More accurate $R-0$-maps would follow naturally from a more accurate spatially-explicit model of tree disease. In this thesis we presented a seasonal $SEIR$-like model 
    that reflected the lifecycle of ADB. However, several simplying assumptions were made when spatially-scaling the small-scale model over GB.
    In particular, we assumed the same level of infectiviy throughout GB. In contrast, ADB severity is known to vary widely in response to the environment.
    Furthermore, in this Thesis infectiviy was treated as a free parameter, and no attempt to fit $\beta$ was undertaken.
    Arguably, fitting infectiviy to ADB mortality data remains one of the highest priority investigations to follow from this Thesis.
    Fitting $\beta$ to ADB mortality data in itself would entail several subsequent model improvements: ash life cycles... forest ecosystems...
    \item Improvements to the model could therefore aid forest restoration, control efforts, and...
    \item We have...Disease cartography using liking species abundance with spatially-explicit epidemic model. A notable challenge posed by a dispersal-based system
    is that regions of land couple together non-locally. Throughout this thesis, CCA assessed clustering based on neighbour links. 
    Going forward, a more sophisticated notion of connectivity could permit higher resolution maps for Inverse power law... method could add significant value... 
    
\end{itemize}