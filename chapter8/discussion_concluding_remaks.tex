
\chapter{Discussion}

In this thesis, a simple model of forest epidemics was incrementally extended into a more elaborate framework.
The incremental improvements led to a novel, general-purpose framework to visualise epidemic severity across GB.
The framework is computationally efficient and adaptable to any wind-dispersed tree pathogen, provided a sufficient host density distribution.
Ultimately, more research is required to progress and test the model against observational disease incidence data.
However, several potentially high-impact research avenues emerge from the work conducted in this thesis\textemdash discussed more below.

After setting the scene with a simple lattice model (SLM) of tree disease in Chapter \ref{chapter:SLM}, 
Chapter \ref{chapter:SLM-applications} linked the SLM with a map of predicted oak abundance given by \cite{hill.data} to produce a large-scale `toy' model.
Consequently, a key result from Chapter \ref{chapter:SLM-applications} demonstrates that nearest-neighbour interactions are
fundamentally insufficient to describe the spread of disease across lower, more realistic landscape tree densities. 
Furthermore, even with non-local dispersal, Chapter \ref{ch5:dispersal-model} demonstrates that an epidemic propagating with a small dispersal length scale gives rise to an unnatural highly wave-like pattern. Together, these results serve to help guide modellers construct more representative models and bolster the growing body of research that highlights the importance of dispersal in botanical epidemiology.

Chapter \ref{ch5:dispersal-model} examined two methods of calculating $R_0$ for a non-local dispersal model (NLM) over a range of epidemic parameters.
Results from Chapter \ref{ch5:dispersal-model} indicate that when the scale of dispersal is comparable to the domain, the model approximates a mass-action well-mixed system described by the standard $SIR$ model. However, comparisons to the $SIR$ model were simplified and limited to one parameter (i.e. the ratio $\beta/\gamma$). Subsequently, the analysis constitutes a preliminary result, and a more sophisticated method of comparison is required to glean further insight.

Most intriguingly, deriving a spatially-explicit analytic expression of $R_0$ led to an `entire function' in Equation \ref{eq:ein}, well-known in complex analysis \cite{abramowitz1948handbook}. 
Equation \ref{eq:ein} contrasts with other non-spatial approaches to calculate $R_0$ that generally rely on infectivity and removal rates alone \cite{...}.
Instead, Equation \ref{eq:ein} is a function of infectivity, removal, dispersal, and tree density parameters.
Further theoretical study could explore a more rigorous analysis of Equation \ref{eq:ein}, or examine an alternate derivation incorporating inverse power law dispersal or exponentially-distributed removal lifetime dynamics.
However, it is clear to see that the mathematical derivation might ultimately become too challenging in the face of more complicated epidemic models. 

Then in Chapter \ref{ch5:dispersal-model}, the analytic expression of $R_0$ was compared against the `actual' contact-traced reproduction ratio. 
Both methods to determine the reproduction ratio demonstrated a clear epidemic threshold at $R_0=1$, thereby marking an important finding.
Nonetheless, comparing both methods revealed that the analytic expression overestimates $R_0$ for progressively higher values ($R_0 \sim 10$),
highlighting an important limitation in the approach. 
Had the analytic expression been applied to dense forest ($\rho >10\%$) where $R_0$ is likely higher, overestimation would only increase.
Therefore, the value of Equation \ref{eq:ein} diminishes rapidly when used to describe highly infectious regimes with a large $R_0$.

In contrast, the contact-tracing method computes the mean number of infections per infected trees over different infected generations.
Computing the mean-generational reproductive ratio proved convenient and led to a sharper epidemic threshold for later generations;
this observations relate nicely to \cite{R0-perc-ref}, who studied a similar method to calculate $R_0$ for foot and mouth disease.
Despite the convenience of contact-tracing the mean number of secondary infections in the model, it would be challenging to experimentally measure in real system.
As such, the method remains applicable to abstract modelling work alone, and not in-the-field experiments.

The central value Chapter \ref{ch:6-adb} results from outlining a novel framework to link a wind-dispersed epidemic model with species abundance data.
In particular, Chapter \ref{ch:6-adb} developed a simplified spatially-explicit $SEIR$ model that described the seasonal spread of ash dieback (ADB) over local spatial scales.
Subsequently, the ADB model was coupled with the map of predicted ash abundance given by \cite{hill.data} to produce an $R_0$-map covering GB.  
Surprisingly, no mechanistic model of ADB could be found in the literature. Thus, to my knowledge, the ADB model presented in this thesis 
marks the first mechanistic compartmentalised attempt at modelling ADB. 

Although ADB has already become established and spread throughout the UK, treating infectiviy as a free parameter in Chapter \ref{ch:6-adb} led to several notable observations.
Particularly, when the infectivity parameter is low, the $R_0$-maps become sparsely-populated with susceptible patches above the epidemic threshold.
Consequently, below a minimum infectivity value, the $R_0$-maps indicate that ADB would be unlikely to invade GB.
In a similar vein, between a narrow range of infectivity parameters clusters in the $R_0$-map grew rapidly, over four orders of magnitude.
Altogether, the model alludes to behaviour akin to a global epidemic phase transition across GB. Unfortunately, however, numerous assumptions place limitations on the framework.

\begin{itemize}
    \item infectivity $\beta$
    \item limited incidence data
    \item small-scale host structure and demography
    \item large-scale between-patch coupling
\end{itemize}








% \begin{itemize}
%     \item Data: modelled abundance data is not accurate enough to make such detailed, fine-scale management decisions, as presented in Chapter 7. According to \cite{hill.data},
%     the RMSE for ash abundance distributions was $5\mathrm{ha}$. Given that connectivity, as we've defined it, can depend on a small number of points,
%     large-inaccuracies could be expected in comparison to the identified areas presented in Chapter 7.
%     \item Until modelled abundance data can identify species such as ash more accurately, future work should concentrate on modelling smaller-scale areas, 
%     where abundance data is known. Country-wide applications are questionable, and extraordinarily ambitious at this point in time. However, several fronts could drastically improve the applicability of the framework presented in this thesis...
%     \item Although landscape management predictions to the individual patch is unrealistic, abundance maps are still expected to capture overall spatial distribution patterns,
%     thus, $R-0$-maps can be expected to present the overarching pattern of pathogen invasiveness.
%     \item More accurate $R-0$-maps would follow naturally from a more accurate spatially-explicit model of tree disease. In this thesis we presented a seasonal $SEIR$-like model \cite{gottwald2002geo}
%     \item we could fit $\beta$ to spatial-temporal tree mortality patterns, "Monte Carlo algorithm using spatiotemporal data from an 18- month epidemiological study in southeast Florida" for citrus canker \cite{neri2014bayesian}. We need biologically plausible parameters
%     \item \cite{WEBIDEMICS}, the risk to outside sources set by the time-to-control, could be factored in to a landscape-level approach
%     that reflected the lifecycle of ADB. However, several simplying assumptions were made when spatially-scaling the small-scale model over GB.
%     In particular, we assumed the same level of infectiviy throughout GB. In contrast, ADB severity is known to vary widely in response to the environment.
%     Furthermore, in this Thesis infectiviy was treated as a free parameter, and no attempt to fit $\beta$ was undertaken.
%     Arguably, fitting infectiviy to ADB mortality data remains one of the highest priority investigations to follow from this Thesis.
%     Fitting $\beta$ to ADB mortality data in itself would entail several subsequent model improvements: ash life cycles... forest ecosystems...
%     \item Improvements to the model could therefore aid forest restoration, control efforts, and...
%     \item We have...Disease cartography using liking species abundance with spatially-explicit epidemic model. A notable challenge posed by a dispersal-based system
%     is that regions of land couple together non-locally. Throughout this thesis, CCA assessed clustering based on neighbour links. 
%     Going forward, a more sophisticated notion of connectivity could permit higher resolution maps for Inverse power law... method could add significant value... 
%     \item focus on a small study area, as in \cite{he2019integrating}
% \end{itemize}

% Interestingly, the disparity between clusters-size, and the distance between clusters could be used to give insight into
% which spatial scale is the dominant driver of disease spread.
% The more sparsely distributed\textemdash and thus fragmented\textemdash the $R_0$-map, the more important LDD becomes as the driver of between-cluster spread.
% Crucially, accessing the relative importance of spatial scale between disease drivers would rely on a fitted value of $\beta$,
% and the knowledge that ADB is estimated to spread with impunity and kill over $85\%$ of ash in Great Britain.


%---------------------------------------------------------
% comment on lack of ADB modelling
% To demonstrate the method, four simplified compartmentalised $SEIR$ models of ADB were constructed, composed of different sporulation and dispersal functions.
% Given the threat ADB poses to European ash, and the resulting economic impact, it was surprising to find a lack of spatio-temporal modelling work in the literature.
% A lack of dynamic spatial-explicit ADB models most likely reflect its complex life-cycle and lack of epidemic parameters.
% However, only in recent years has the sexual reproductive mode become widely accepted as the dominant disease-driver \cite{https://doi.org/10.1111/ppa.12844, havnavckova2017direct}.
% In addition to the life-cycle of ADB, multi-scale parameters for ascospore dispersal have only been published within the last few years \cite{grosdidier2018tracking}.

% To define invasibility over the landscape, we relied on a spatially-explicit value of $R_0$\textemdash i.e the mean number of infections produced by the first generation of infected ash.

%---------------------------------------------------------
% relation of work to more commonplace Metapopulation models
% In the process of computing a value for $R_0$, several insights into dispersal-based epidemic progression and spatial-scale came to light.
% Namely, computing $R_0$ for a fat-tailed inverse power law dispersal kernel required a larger to domain-size in comparison to the thin-tailed Gaussian variants.
% A larger domain-size necessitated a lower landscape-level resolution (or equivalently larger patch-size) when forming the $R_0$-map.
% Thus to define an $R_0$-value in the framework, landscape-level patch-size needs to reflect the nature of dispersal;
% metapopulation-type models do not have this limitation. 
% Yet, at the same time typical metapopulation models are not predicated on small-scale epidemiological properties, 
% but tend to model patch-sizes on the order of $100\mathrm{m}-1\mathrm{km}$ \cite{large-scale-control, doi:10.1111/j.1365-3059.2010.02391.x}.

%---------------------------------------------------------
% generalisation to the approach
% In favour of parsimony, infectivity $\beta$ was kept constant over the entire landscape of Great Britain.
% However, epidemic severity of ADB is known to vary in response to either urban or rural environments \cite{marciulyniene2017can}.
% So in reality, $\beta$ may very well exhibit a spatial dependence.
% Likewise, including an index on, $\beta_i$, could reflect yearly changes due to climate change or habitat suitability\textemdash thus generalising the yearly infection cycles according to Figure \ref{fig:SEIR-transitions}.
% An attractive feature of the $SEIR$ model therefore includes the scope of its adaptability 
% and generalisations to our notion of $\beta$ could support more in-depth studies on spatio-temporal epidemic heterogeneity.

%-------------------------------------------------------
% - Crucially, future work will involve integrating LDD mechanisms into the model in order to understand the relative importance long vs local distance dispersal. We may speculate about the relative importance looking at figure x, whereby the maximum distance spread in season due to local-scale spread is xm/year, in stark contrast from the observed spread of 40-60km/yr.
% - We cannot overstate the importance of LDD, and it is hard to say the degree to which targeting the local dispersal mechanism alone will inhibit the spread. We will revisit this question in future work, however, we contend that preferentially targeting diseased trees based on spatial location.....could help control epidemics with greater efficacy. 
% - We may speculate how our result could aid the effort of choosing where to re-plant ash stands genetically engineered to be less susceptible; if re-planting efforts were undertaken in certain location.... <- speculative
% - We may speculate about how persistent ash dieback would be, even if a large-scale control effort was undertaken
% - There is evidence to suggest regional variation in mortality due to ash dieback \cite{stocks2017first}, this could be incorporated into the model...
% - Recently, it has been suggested that the dispersal-kernel of wind-borne pathogens might follow a scaling law \cite{https://doi.org/10.1111/jbi.13642}


% \section{Patch-to-patch transmission}
% \label{sec:cast-study-jump-patches}

% \textcolor{red}{
% \begin{itemize}
%     \item \textbf{Outline:}  Here I intend to ascertain the probability of patch-to-patch transmission as a function of distance
%     \item \textbf{Hypothesis:} I envision that the Gaussian-based model will not disperse beyond $\sim 0.70\mathrm{km}$ for any level of infectivity. 
%     On the other hand, inverse power law spread will disperse up to, and beyond, $10 \mathrm{km}$ but the spread beyond $5 \mathrm{km}$ will become rare.
%     \item \textbf{Motivation:}  Accessing an upper bound for both models for different levels of $\beta$ should provide two different length-scales
%     and motivate density reductions in the next section
%     \item \textbf{Time est:} $\leq 1$ week
% \end{itemize}
% }

% \section{Intermediately density-reductions}
% \label{sec:density-reductions}

% \textcolor{red}{
% \begin{itemize}
%     \item \textbf{Outline:} Here, I will try to gauge how density reductions disrupt the spread of disease in a coupled system of patches. 
%     That is, I will consider three nearest-neighbour patches e.g. a source, target and control. The source patch will host the infection at 
%     $t=0$, and the I will examine how spread to the target patch evolves in response to different levels of control.
%     \item \textbf{Hypothesis:} we can control the Gaussian-based spread with relative ease, and we can slow the inverse-power law spread with more severe density reductions
%     \item  \textbf{Motivation: } assessing the patch-to-patch transmission gives a stronger argument for introducing the containment strategy.
%     \item \textbf{Time est:} $\leq 2$ weeks
% \end{itemize}}