% ------------------------------------------------------------------------------
% A primer on continuum models and how this relates to the work we have done
% 1) Introduce logistic growth and diffusion
% 2) Introduce FKPP in all its glory .
% 3) Show one and two d spread
% 4) Couple to the sub-grid, what this says about the sub-grid...
% 5) Talk about growth rate-alpha and growth rate of sub-grid... an assumption about mixing and SIR <--         appendix territory
% 6) Show `toy model properties in alpha and d' get realistic looking spread...
% 7) Show realistic spread and give a primer to things which could be done.
% 8) Talk about alternate more, non-linear models and consult Rammile...
% ------------------------------------------------------------------------------

\chapter{\textcolor{red}{Towards landscape-level control (WORK IN PROGRESS)}}

There has been a great deal of work carried out into the nature of control in plant and tree-based epidemics\footnote{See section \ref{chapter2:plant-ecologoy} for a review on the control in plant-based epidemics.}. In particular, the spatial structure of plant-hosts is an essential factor when considering how to manage an outbreak \cite{spatial-control-optimisation, control-heterogeneous-landscapes}. The accepted paradigm of control typically considers infected tree removals over a relatively small spatial scale, near infected hosts \cite{WEBIDEMICS}, or more broadly, ahead of the wavefront \cite{large-scale-control}. However, landscape-level epidemic control, based solely on the structure of large-scale spatial distribution of hosts incorporating topography, has yet to be explored in-depth.\\

As such, in this chapter, we will examine how host-heterogeneity, under the influence of a wind-dispersed pathogen, can give rise to natural pinch-points and fault lines in the spatial distribution of hosts. Population pinch points may give rise to a bottleneck in the epidemic spread, which in principle, may be exploited with targeted tree felling to fragment the host population with minimised effort. In essence, a strategy of 'regional containment', targeting the local wind-based pathogen dispersal mechanism, is formulated and scaled up over large spatial scales. Similar concepts for crop and livestock diseases have been outlined \cite{PAPAIX201435, GILIOLI20131, Gilligan-disease-management}, however, to our knowledge, this has not been generalised to tree population distributions over large spatial scales.

% - This leads us to develop a heuristically-based fragmentation algorithm. 
% -As we define it, fragmentation considers which locations in the population, if artificially taken below $R_0 = 1$ through felling, would disrupt epidemiological connectivity\textemdash, thus leading to containment. 
% -Epidemic containment in the largest $R_0$-cluster is then analysed and shown to be most applicable over a specific range of infectivity parameters.
% -Although the strategy of control presented in this chapter is demonstrated on a simplified model of ADB, the results are generic and could be applied to any wind-dispersed pathogen.
% - Developing a landscape-level control strategy when there is LDD (and epidemic uncertainty) present several obstacles, 
% - The emergent epidemic caused by the pathosystem ash dieback (ADB) is predicted to wipe out the vast majority of ash in Great Britain over the next few decades \cite{ash-dieback-costs}. 
% - Currently, large-scale control efforts aim to slow the spread, in contrast, to complete containment. % Find references of current efforts/guidelines 
% - A slower rate of spread benefits ash populations allowing them to recover alongside artificial replanting. % reference 

% - The challenge of controlling ADB reflects the challenge of containing a pathogen that spreads via long-distance dispersal (LDD). 
% Moreover, HP can infect ash through diverse mechanisms such as water-course and contaminated soil and LDD means that new and distant foci can emerge over large distances without the need for nearby ash\textemdash for a more detailed review of ADB, including the challenges of control and biology, see chapter \ref{chapter2:litrevieiw}.
% Subsequently, we found connectedness inside a given cluster could depend on just a small number of `connecting points' which if removed, thinned below $\rho_c$, would lead to significant fragmentation and divide the cluster

% Suppose two patches of land, having tree densities above the threshold, $R_0 = 1$, are separated by an intermediary patch below the threshold. If we are not careful, dispersal could traverse the below-threshold patch within a single jump, even at local spatial scales. As such, we must choose a domain resolution that gives some assurance that, at the local scale, wind-dispersed spores cannot jump over whole patches with singular jumps. As always, there is the possibility that spores disperse more considerable distances, from mainland Europe to Great Britain, for example, \cite{freer2017tree, wylder2018evidence}. Nevertheless, this chapter aims at targeting dispersal at local scales, and LDD can be omitted for now.


\section{Optimal fragmentation of $R_0$-maps}

Each cluster (denoted by $\mathbf{C}$) detected in the $R_0$-map represents a connected network of susceptible Ash where pathogen survival and spread is possible. The shape of each cluster is constrained by landscape topography and geography. If no control is attempted, all trees within a cluster are put at risk if one point in the cluster becomes infected. Given this, a control strategy is formulated by noting that removing specific positions of Ash (or breaking critical `links') via selective tree felling would efficiently break the cluster into two fragments. A technical explanation of the fragmentation algorithm is given in Appendix B.\\

Cluster fragments define two disconnected `sub-clusters' (named $\mathbf{C_1}$ and $\mathbf{C_2}$). This eliminates risk for trees inside one sub-cluster and accomplishes control by regionally containing an epidemic inside a `confining cluster'. Ash trees inside the confining cluster are assumed to be removed by the pathogen while Ash trees inside the remaining sub-cluster survive and remain susceptible. If the number of felled trees is low in comparison to the number saved, then efficient control is achieved.\\

The notion of fragmenting a cluster $\mathbf{C}$ into two sub-clusters $\mathbf{C_1}$ and $\mathbf{C_2}$ may be repeated $N$ times to produce a set of disconnected sub-clusters. After each fragmentation, sub-clusters were ranked according to the Ash population size they contained. This allowed the largest sub-cluster to be targeted in the next iteration. \textcolor{red}{Figure X} shows the largest cluster identified in the $R_0$-map (named $\mathbf{C_T}$) iteratively fragmented $N=5$ times culminating in six disconnected sub-clusters, depicted as the colour-filled regions from orange-green. The zoomed inset of $\mathbf{C_T}$ highlights the critically-connecting links found during each fragmentation.\\


We assume the average Ash tree in the population covers an area of $\mathrm{25m^2}$ (giving a maximum of $400$ trees per hectare of canopy cover \cite{ash-tree2, ash-tree1}). Combining this assumption with the data permits us to estimate the number density of Ash trees per point and subsequently the population size of Ash contained inside each sub-cluster. Additionally, an estimate towards the number of felled trees can be formed, thus giving insight into the scale of resource expenditure. This is achieved by first finding the difference between $\rho_c$ and the density values of critical links (i.e. the level of required density reductions), then multiplying this by the number density.\\

\textcolor{red}{Figure \ref{fig:result-cluster-reductions}(b)} shows the maximum sub-cluster population size in blue alongside the estimated number of trees needed to be felled in red. The largest sub-cluster continually decreased over $N=30$ iterations of fragmentation. Size reductions occurred more rapidly at first, starting from $\mathrm{3.5\times 10^7}$ trees and decreasing to $\mathrm{1.1\times 10^7}$ trees within $5$ iterations. By $N=30$, the rate of sub-cluster size reductions leveled off suggesting fragmentation in the field may tend towards diminishing returns on control efficiency. That is, larger values of $N$ produce smaller changes in the number of saved trees in proportion to the expenditure of resources needed to fell trees. This is demonstrated when the red and blue lines begin to approach each other, attaining comparable values.\\ 

\textcolor{red}{Figure \ref{fig:result-cluster-reductions}(c)}, sub-cluster size reductions in $\mathbf{C_T}$ are shown alongside the $2^{nd}$ and $3^{rd}$ largest $R_0$-clusters on a log-log plot\textemdash from solid blue through to green. The straight lines indicate a power-law. The sub-cluster size reductions were fitted to a power-law of the form $f(x) = ax^{-k}$, shown by the corresponding dashed lines. Fitted parameter values of $a$ and $k$ represent the initial cluster population size and rate of decrease respectively. Importantly, all sub-cluster size reductions demonstrate an exponent of $k\sim 3/4$. In addition to the Ash species shown here, Conifer (\textit{Pseudotsuga menziesii}) and Beech (\textit{Fagus sylvatica}) were given by \cite{hill.data} and were used comparatively to test fragmentation. For each species considered, all iteratively fragmented clusters had similar exponent values, $k\sim 3/4$.\\


\subsection{Towards epidemic control}
% Main results figure 7
Regional containment as a strategy of epidemic control was tested by considering outbreaks from different epicenters inside the target cluster $\mathbf{C_T}$. Starting from an epicenter, epidemic containment can be achieved in a variety of ways. \textcolor{red}{Figure \ref{fig:scenario-expo}} demonstrates this for a single epicenter marked by the black cross. The number of fragmentation iterations is varied from $N=1$ to $N=30$. For each step $N$, we identify which critical links, shown in red and denoted as $F$, should be felled below $\rho_c$ in order to define a confining sub-cluster around the epicenter. The population of saved Ash, illustrated in light grey, remains in state $S$ while all trees inside the confining sub-cluster, shown in dark grey, are assumed to transition into state $R$. By assessing the number of trees saved against the number of trees felled we can define a `payoff' ratio as $N_S/N_F$.\\

A set of epicenters were defined in $\mathbf{C_T}$ (by identifying the center of mass for each sub-cluster when $\mathbf{C_T}$ was iteratively fragmented $N=30$ times). From $31$ different epicenters and $30$ iterations, $930$ containment scenarios were simulated. For some epicenters, typically in close proximity, containment looked identical up to $N$ iterations before a different payoff ratio was registered. Subsequently, less than $930$ unique data-points between $N_S$ and $N_F$ were found\textemdash shown in \textcolor{red}{Figure\ref{fig:result-culling-efficiency}}.\\

The payoff ratios determined from the top $50$ performing containment scenarios were then ranked and plotted, shown in \textcolor{red}{Figure} \ref{fig:result-culling-efficiency}(a) by the blue line overlaid with a coloured scatter plot. For the purposes of our model, the payoff efficiency is shown alongside the corresponding number of felled trees in dashed grey. Payoff efficiencies begin to level off around $\mathrm{10^3}$ involving around $\mathrm{10^4}$ felled trees. In reality, this would be a challenge to accomplish in a reasonable time-frame. \textcolor{red}{Figure}\ref{fig:result-culling-efficiency}(b) shows a scatter plot of all the data, $N_S$ and $N_F$ are plotted with color corresponding to the payoff. The efficiency ranged over multiple orders of magnitude up to a maximum efficiency of $\mathrm{10^6}$. The top left quadrant of \textcolor{red}{Figure} \ref{fig:result-culling-efficiency}(b) represents viable scenarios where containment felling could be pragmatically implemented by policy makers with the greatest efficiency.\\

A small number of exceedingly high payoff results were found. In \textcolor{red}{Figure}\ref{fig:result-culling-efficiency}(c), the top three ranking payoff scenarios are illustrated on the same map and reinforces the intuitive notion that epicenters around edge positions can be most efficiently contained. The zoomed inset highlights just $\mathrm{2\times 1km^2}$ Ash patches located at a single point of contact for the top ranked sub-cluster. Critical links for the best performing scenarios were all found in lower-density regions within the $R_0$-cluster and were easily fragmented with a low number of felled trees. This highlights the possibility that fragmentation is most easily achieved with tree felling through critically linking regions and demonstrates how regional variation in host density can be exploited for efficient control. 


% Notes
% A quantity of interest that appears frequently in the literature is the initial growth rate $r$, that is the density-independent growth at the start of any epidemic <- this could be linked with alpha


\section{Chapter summary}

The article published by \cite{time-varying-infectivity} indicates the possibility of \textit{preferentially} controlling an area based on the final-sized epidemic. It goes without saying that areas of land that have the largest final sized epidemic are likely the most density populated. However, we outline the possibility it might be more beneficial to preferentially control an area based on its spatial location and how it couples to to neighbouring areas.

% genetic tolerance is currently the only viable strategy of control\cite{kosawang2018fungal}

% Crucially, future work will involve integrating LDD mechanisms into the model in order to understand the relative importance long vs local distance dispersal. We may speculate about the relative importance looking at figure x, whereby the maximum distance spread in season due to local-scale spread is xm/year, in stark contrast from the observed spread of 40-60km/yr.

% We cannot overstate the importance of LDD, and it is hard to say the degree to which targeting the local dispersal mechanism alone will inhibit the spread. We will revisit this question in future work, however, we contend that preferentially targeting diseased trees based on spatial location.....could help control epidemics with greater efficacy. 

% We may speculate how our result could aid the effort of choosing where to re-plant ash stands genetically engineered to be less susceptible; if re-planting efforts were undertaken in certain location.... <- speculative

% We may speculate about how persistent ash dieback would be, even if a large-scale control effort was undertaken

% There is evidence to suggest regional variation in mortality due to ash dieback \cite{stocks2017first}, this could be incorporated into the model...

% Our results support the call for more research to be undertaken into multi-scale dispersal, 

% Recently, it has been suggested that the dispersal-kernel of wind-borne pathogens might follow a scaling law \cite{https://doi.org/10.1111/jbi.13642}. The significance of such a finding would allow us to analyse the $R_0$-maps over much more flexible spatial scales. % Explain.

% This sentence is wrong, the cited paper makes an argument for the spatially-scaling up of dispersal kernels, which happens to still be useful paper to cite, re-phrase and re-frame accordingly.

% see \cite{ash-dieback-costs}, and the references therein (methodology excel spread sheet S1), for mortality references the latest findings suggest a mean mortality rate of 95%.
