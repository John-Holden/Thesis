
% ------------------------------------------------------------------------------
% Introduction
% Set the scene:
% 1) What is this thesis ? What did we find ? 
% 2) Where are gaps in the filed and what we contributed ? 
% 3) Why is this important and why mathematical modelling is a solution ? 
%.   - main parameters considered and prelude to main assumptions of the model.
% 4) The thesis summary, highlight key findings and where we sit in the field ? 
%.   - what the thesis entails and the general story line.
% -----------------------------------------------------------------------------
\chapter{Introduction}

% \begin{itemize}
%     \item talk about various routes to understanding tree disease Fig 2 of \cite{pub.1012384986} \cite{francl2001disease}
%     \item reference the applications of a model, predict the epidemic, predict control measures, crop loss, invasion and persistence, the risk of disease etc..
%     \item \cite{disease-biodiversity} disease key in biodiversity..., \cite{JOHNSON2002129} importance for landscape integrity
%     \item \cite{kelly2002monitoring} technology and capturing data via satellite. RS application \cite{doi:10.1094/PHYTO.2003.93.12.1524} \cite{doi:10.1080/0143116031000139926}
%     \item \cite{doi:10.1046/j.1523-1739.1994.08010256.x} fragmentation reference...
%     \item insects and pests intelligently seek out hosts, spores and fungicides occur passively, \textcolor{red}{double check}
%     \item talk about local infection and resolution of the problem, e.g. complicated biology of the pathogen vs dispersal distance
%     \item talk about the host-pathogen-environment dynamic \cite{pub.1012384986} Fig 3, how we are concentrating on the modelling perspective.
%     \item reference the scale problem \cite{van1999pandemics} only used small amounts of detail for large scale spread
% \end{itemize}

Modernity has witnessed drastic increases in the severity of tree disease epidemics. 
The recent development of international trade, climate change, and monocultures have increased the risk of large-scale
outbreaks in native plant and tree populations. This introductory chapter informs the reader about the importance, policy,
and modelling of infectious tree disease epidemics.

First, the motivation for studying tree disease epidemics is highlighted, 
following a review on the critical drivers of tree disease epidemics. 
Then, the challenge of evaluating epidemic control strategies, and the subsequent implementation from policymakers, is surveyed.
This Thesis will focus on the mathematical and computational modelling of tree-based epidemics in Great Britain (GB).

\newpage

\section{Invasive tree diseases}

The modern world relies heavily on imports and exports characterised by global trade networks. 
Unfortunately, importing and exporting foreign plant material can introduce invasive pests and pathogens
into non-native landscapes. Consequently, this poses worldwide risks to crops, flowering plants
and trees that may lack evolutionary defences to invasive species. \cite{doi:10.1002/9781444329988.ch8}. 

Epidemics through plant populations can be devastating.
Classic examples include Irish potato blight, Dutch elm disease \cite{doi:10.1111/j.1365-3059.2010.02391.x} 
and North American chestnut blight \cite{doi:10.1002/9780470535486.ch7}.
Two epidemics currently underway in the UK include ash dieback (Hymenoscyphus fraxineus) affecting European ash trees (Fraxinus excelsior)
\cite{ash-dieback-costs}, and Phytophthora ramorum, a prevalent disease that affects over 150 plant species, including oak, 
larch, and sweet chestnut \cite{p.ramourum}.

Trees play a pivotal role in terrestrial ecosystems \cite{boyd2013consequence}, 
and ensuring tree health marks a pressing challenge for society.
The most widely known and well-researched motivates for studying tree disease epidemics include economic,
climatic and ecological function \cite{ash-dieback-costs, freer2017tree, boyd2013consequence, tyrvainen2005benefits}. 
In particular, policymakers can execute a variety of strategies to impede the spread of disease 
to meet this complicated challenge \cite{pests-intro, Gilligan-disease-management}. 
Thinning host densities \cite{resiliency-density-reductions} or planting genetically diverse configurations
\cite{doi:10.1094/PD-89-0969, genetic-heterogeneity, huang1980importance} have been shown to increase
pathogen resilience. 

% - The progression of an epidemic can be briefly summarised by the time-line of, arrival, spread, impact and management.
% \textcolor{blue}{Expand this to include some numbers on cost and societal effects. Go into more examples of how tree diseases can cause harm to ecosystems and expand on wider environmental impacts e.g. Carbon sequestration, soil bindings and biodiversity. Expand  on current threats. !!! Talk about biology of pathogens and main divisions of diseases !!!}

% \begin{itemize}
%     \item \textcolor{red}{I could expand on the economics, along with climatic and ecological importance of trees}
% \end{itemize}

% Trees grow naturally, in both rural and urban settings, this includes commercially managed plantations and orchards. 
% Protecting and ensuring tree-health is of essential importance for society \cite{Boyd1235773}. 

% \begin{table}
%     \centering
%     \begin{tabular}{|p{3cm}||p{13cm}| }
%     \hline
%     \textbf{Benefit}&\multicolumn{1}{c|}{}\\
%     \hline
%     Social  & Recreational activities, mental and physical health, cultural and historic sentiment.\\
%     \hline
%      Aesthetic & Landscape variation, textures, colours. Seasonal dynamics which change landscape views. \\
%     \hline
%      Climatic &  Cooling, wind control, impacts on urban climate through temperature and humidity control. Air pollution reduction, sound control, glare and reflection on reduction, flood prevention and erosion control. \\
%     \hline
%      Ecological & Biodiversity and biotopes for flora and fauna.\\
%     \hline
%      Economic & Timber, wood pulp, fiber and food.\\
%     \hline
%     \end{tabular}
%     \caption{The benefits of tree health, based on \cite{tyrvainen2005benefits} and \cite{boyd2013consequence}}
%     \label{table:tree-health}
% \end{table}


\subsection{Tree disease drivers}

%-Talk about the different types of interactions and drivers, based on the pathogen biology and type of disease. How might one conceptualise it mathematically and what are the main factors to consider with tree diseases. Give an intuitive justification of the parameters used in the modelling work. Density infectivity and dispersal

The terms `pest' and `pathogen' denote a broad spectrum of taxonomically diverse organisms. 
Pests represent an organism that harms humans or human interests such as crops or livestock
\cite{buckle2015rodent, oerke2006crop, de1964biological}. Overwhelmingly, insects constitute the main
pest threats to tree species \cite{metcalf1994introduction}. In Great Britain (GB), Asian longhorn beetle
(ALB) \cite{haack2010managing}, and oak processionary moth OPM \cite{tomlinson2015managing} are two concerts
that currently threaten tree health.

In contrast, the term `pathogen' denotes any organism that induces disease. 
In the context of trees populations, diseases include fungi, bacteria, viruses, and oomycetes \cite{balloux2017q, Boyd1235773}. 
Currently, the oomycete \textit{Phytophthora ramorum} \cite{brasier2005phytophthora}, and Ash dieback (ADB)
caused by the fungus \textit{Hymenoscyphus fraxineus} \cite{ash-dieback-costs, mitchell2014ash} are two pathogens
that threaten tree-health in Great Britain.

% \item \textcolor{red}{Expand on the plant-passport, how do I cite government reports ?}

The trade and transport of foreign plant material are widely recognised to increase the risk
of introducing pests and pathogens into non-native landscapes \cite{POTTER201761, lovett2016nonnative, roy2014increasing}.
Epidemics caused by non-native pathogens can be catastrophic to tree populations that lack immunity and genetic
resistance to the invasive species \cite{doi:10.1002/9781444329988.ch8}; this can be understood from an evolutionary perspective: 
in an environment unaltered by human transportation, tree and plant species are thought to co-evolve alongside pests and pathogens
in a gene-for-gene like arm-race \cite{flor1971current, dangl2001plant, Thrall1735}. 

However, the introduction of a foreign
pathogen can overwhelm a population that has no such immunity \cite{desprez2016evolutionary}. Two classic examples that shook
the world are: Dutch elm disease \cite{doi:10.1111/j.1365-3059.2010.02391.x} in the United Kingdom and chestnut blight
\cite{doi:10.1002/9780470535486.ch7} in North America.

% Control of tree disease
Recently, the importance of effective trade regulations for preventative epidemic control has become apparent
\cite{rodoni2009role}. The role of shipping and human transport is an essential factor that risks the introduction
of invasive pests and pathogens into vulnerable landscapes within a country. For example, the shipping of elm timber
infected with scolytid bark beetles, carrying the fungus \textit{Ophiostoma novo‐ulmi} was identified as an essential
factor in driving the Dutch elm outbreak in the United Kingdom \cite{doi:10.1111/j.1365-3059.2010.02391.x}. 

Thus, adequate border controls are an indispensable step in a nations arsenal to stop the spread of disease.
Ordinarily, preventative measures take the form of customs checks on imported and exported plant material such
as timber, crops or horticultural goods. In particular, the European Commission enacted plant passports to regulate
how growers and traders can transport plant material between countries\footnote{In light of Brexit, the UK now plans
to implement an equivalent passport} \cite{wulfert2010implementation}.

If checks and policy implementations fail, a pathogen might be introduced into the landscape and spread through natural
dispersal pathways. At this point, biological control becomes necessary. The biological control of plant-based disease
can be achieved in numerous ways. Commonly, this involves chemical agents such as pesticides, predatory insects or planting
genetically resistant cultivars \cite{pal2006biological, baker1974biological}. 

In this thesis, we are motivated to investigate the eradication of tree-based pathogens. 
Here, we are typically limited to eradicating infected and diseased trees through sanitation felling. 
In this case, the questions to be answered are A) How do we effectively identify an infected tree? B) 
Which infected trees are the best choices to fell? C) What is the risk that a large-scale epidemic will result?


\section{Modelling and policy}

% \cite{thompson2016management}  - problems associated with 'control'
% \cite{gaydos2019forecasting} -  workshop with policy makers
% \cite{jones2020modelling} - developing plant health models in conjunction with decision makers 
% \cite{tsouvalis2019post} - ash dieback politics

The benefit of controlling an epidemic should outweigh the costs of letting an outbreak spread unchecked. 
Plant disease modellers can help infer well-designed control policies that maximally reduce epidemic impact and minimise the expenditure
of resources\textemdash both natural and economic. However, achieving this in practice is problematic due to various unknowns \cite{13-challenges}.
Moreover, history gives examples of insufficient control policies which fail to halt pathogen spread. 
Prominent examples include Dutch elm disease in the late 1960s and early 1970s \cite{dutch-elm-mismanage}, 
and more recently citrus canker in Florida \cite{schubert2001meeting}.

With mathematical models we may attempt to understand what dictates optimal control of tree diseases. 
Strategies have been explored on both smaller landscapes \cite{WEBIDEMICS, risk-potential-control} 
and larger landscapes \cite{large-scale-control, large-scale-control2}. Current consensus on all spatial scales
agree that the scale of response must equal the scale of epidemic \cite{control-scale-matching}. Furthermore, 
any response must be carried out swiftly otherwise the likelihood of successful management rapidly decreases and the cost of inaction soars.

Conventional eradication strategies involve detecting symptomatic trees and culling neighbours within a radius \cite{WEBIDEMICS}. This is made difficult by numerous factors including the cryptic nature of tree diseases and resource constraints which may vary over time \cite{control-theory, control-theory-application}. The naive strategy can be fine-tuned and optimised in many ways to increase efficiency. One scheme involves ranking targets according to the risk they pose in order to prioritise culling \cite{risk-potential-control}. Over large-scales, evidence suggests epidemics are most effectively controlled by targeting infected trees either at or ahead of the infectious wave-front, as has been shown for sudden oak death in California \cite{large-scale-control}. In this study, optimal tree eradication is developed by considering which locations ahead of the wave-front make for efficient targets when taking into account large-scale host structure.


\section{Chapter summary}

% \begin{itemize}
%     \item Talk about the general methods and strategies favoured in this thesis. Discuss main findings and the layout of this thesis, discuss where the main results fit in i.e. control of diseases.
% \end{itemize}

The main aim of this thesis was to develop workable models of infectious tree diseases from the ground up, from which, robust and more refined models can be established. Developing a simple model of infectious tree diseases was found to have merits when applied to realistic data sets of hosts. The aim was to further understand the nature of epidemics through tree populations. The research questions of this project involved asking to what extent a simple lattice model need be extended and what assumptions were important for each conceptual leap in modelling. The modelling work was in conjunction with .... and aimed to establish a starting model.

From a simple-localised lattice model of tree disease a sub-grid method was developed and found to be applicable to large-realistic data sets. The thesis contribute a novel, intuitive notion of control based based on host-spatial structure, a multi-scale approach to modelling tree disease (the sub-grid) and investigated a simple, tractable model on a realistic data-set which was found to give insight into the nature of disease spread on heterogeneous data. These results and ideas are generalisable to many instances of tree diseases.
