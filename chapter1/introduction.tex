
% ------------------------------------------------------------------------------
% Introduction
% Set the scene:
% 1) What is this thesis ? What did we find ? 
% 2) Where are gaps in the filed and what we contributed ? 
% 3) Why is this important and why mathematical modelling is a solution ? 
%.   - main parameters considered and prelude to main assumptions of the model.
% 4) The thesis summary, highlight key findings and where we sit in the field ? 
%.   - what the thesis entails and the general story line.
% -----------------------------------------------------------------------------
\chapter{Introduction}

% \begin{itemize}
%     \item talk about various routes to understanding tree disease Fig 2 of \cite{pub.1012384986} \cite{francl2001disease}
%     \item reference the applications of a model, predict the epidemic, predict control measures, crop loss, invasion and persistence, the risk of disease etc..
%     \item \cite{disease-biodiversity} disease key in biodiversity..., \cite{JOHNSON2002129} importance for landscape integrity
%     \item \cite{kelly2002monitoring} technology and capturing data via satellite. RS application \cite{doi:10.1094/PHYTO.2003.93.12.1524} \cite{doi:10.1080/0143116031000139926}
%     \item \cite{doi:10.1046/j.1523-1739.1994.08010256.x} fragmentation reference...
%     \item insects and pests intelligently seek out hosts, spores and fungicides occur passively, \textcolor{red}{double check}
%     \item talk about local infection and resolution of the problem, e.g. complicated biology of the pathogen vs dispersal distance
%     \item talk about the host-pathogen-environment dynamic \cite{pub.1012384986} Fig 3, how we are concentrating on the modelling perspective.
%     \item reference the scale problem \cite{van1999pandemics} only used small amounts of detail for large scale spread
% \end{itemize}

Currently, tree health is threatened worldwide on numerous fronts.  
Drastic increases in international trade, climate change, and the widespread adoption of 
monocultures have increased the risk of large-scale outbreaks in native tree populations. 
Although, scientists, policymakers and stakeholders can cooperate to prevent the spread of tree disease epidemics. 
In particular, epidemic models can be used to help design effective strategies to prevent the spread of tree disease 
through rural and urban environments. 

However, a plethora of challenges complicate effective on-the-ground responses. 
For example, epidemic drivers are multifaceted, hard to quantify, and often unknown. 
Subsequently, mathematical models are significantly challenging to parameterise. Moreover, communicating
highly technical research insights to policymakers and stakeholders pose a considerable challenge
to modellers even with accurate parameterisation. 

This introduction recounts high-level background information on tree health and the interactions
between mathematical modellers and policymakers.
First, the motivation for studying tree disease epidemics is highlighted, 
followed by a review of fundamental tree disease drivers.
Then, vital interactions between policymakers, modellers and stakeholders are highlighted alongside the
the challenge of implementing epidemic control.

\newpage

\section{Invasive tree diseases}

The modern world relies heavily on imports and exports characterised by global trade networks. 
Unfortunately, importing and exporting foreign plant material can introduce invasive pests and pathogens
into non-native landscapes. Consequently, this poses worldwide risks to crops, flowering plants
and trees that may lack evolutionary defences to invasive species. \cite{doi:10.1002/9781444329988.ch8}. 

Epidemics through plant populations can be devastating.
Classic examples include Irish potato blight, Dutch elm disease \cite{doi:10.1111/j.1365-3059.2010.02391.x} 
and North American chestnut blight \cite{doi:10.1002/9780470535486.ch7}.
Two epidemics currently underway in the UK include ash dieback (Hymenoscyphus fraxineus) affecting European ash trees (Fraxinus excelsior)
\cite{ash-dieback-costs}, and Phytophthora ramorum, a prevalent disease that affects over 150 plant species, including oak, 
larch, and sweet chestnut \cite{p.ramourum}.

Trees play a pivotal role in terrestrial ecosystems \cite{boyd2013consequence}, 
and ensuring tree health marks a pressing challenge for society.
The most widely known and well-researched motivates for studying tree disease epidemics include economic,
climatic and ecological function \cite{ash-dieback-costs, freer2017tree, boyd2013consequence, tyrvainen2005benefits}. 
In particular, policymakers can lead a variety of control initiatives to impede the spread of disease 
to meet this complicated challenge \cite{pests-intro, Gilligan-disease-management}. 
As an example, thinning host densities \cite{resiliency-density-reductions} or planting genetically diverse configurations
\cite{doi:10.1094/PD-89-0969, genetic-heterogeneity, huang1980importance} have been shown to increase
pathogen resilience. 

% - The progression of an epidemic can be briefly summarised by the time-line of, arrival, spread, impact and management.
% \textcolor{blue}{Expand this to include some numbers on cost and societal effects. Go into more examples of how tree diseases can cause harm to ecosystems and expand on wider environmental impacts e.g. Carbon sequestration, soil bindings and biodiversity. Expand  on current threats. !!! Talk about biology of pathogens and main divisions of diseases !!!}

% \begin{itemize}
%     \item \textcolor{red}{I could expand on the economics, along with climatic and ecological importance of trees}
% \end{itemize}

% Trees grow naturally, in both rural and urban settings, this includes commercially managed plantations and orchards. 
% Protecting and ensuring tree-health is of essential importance for society \cite{Boyd1235773}. 

% \begin{table}
%     \centering
%     \begin{tabular}{|p{3cm}||p{13cm}| }
%     \hline
%     \textbf{Benefit}&\multicolumn{1}{c|}{}\\
%     \hline
%     Social  & Recreational activities, mental and physical health, cultural and historic sentiment.\\
%     \hline
%      Aesthetic & Landscape variation, textures, colours. Seasonal dynamics which change landscape views. \\
%     \hline
%      Climatic &  Cooling, wind control, impacts on urban climate through temperature and humidity control. Air pollution reduction, sound control, glare and reflection on reduction, flood prevention and erosion control. \\
%     \hline
%      Ecological & Biodiversity and biotopes for flora and fauna.\\
%     \hline
%      Economic & Timber, wood pulp, fiber and food.\\
%     \hline
%     \end{tabular}
%     \caption{The benefits of tree health, based on \cite{tyrvainen2005benefits} and \cite{boyd2013consequence}}
%     \label{table:tree-health}
% \end{table}


\subsection{Tree disease drivers}

%-Talk about the different types of interactions and drivers, based on the pathogen biology and type of disease. How might one conceptualise it mathematically and what are the main factors to consider with tree diseases. Give an intuitive justification of the parameters used in the modelling work. Density infectivity and dispersal

The terms `pest' and `pathogen' denote a broad spectrum of taxonomically diverse organisms. 
Pests denote any organism that harms humans or human interests such as crops or livestock
\cite{buckle2015rodent, oerke2006crop, de1964biological}. Overwhelmingly, insects constitute the main
pest threats to tree species \cite{metcalf1994introduction}. In Great Britain (GB), Asian longhorn beetle
(ALB) \cite{haack2010managing}, and oak processionary moth OPM \cite{tomlinson2015managing} are two concerts
that currently threaten tree health.

In contrast, the term `pathogen' denotes any organism that induces disease. 
In the context of trees populations, diseases include fungi, bacteria, viruses, and oomycetes \cite{balloux2017q, Boyd1235773}. 
Currently, the oomycete \textit{Phytophthora ramorum} \cite{brasier2005phytophthora}, and Ash dieback (ADB)
caused by the fungus \textit{Hymenoscyphus fraxineus} \cite{ash-dieback-costs, mitchell2014ash} are two pathogens
that threaten tree-health in GB.

The trade and transport of foreign plant material are widely recognised to increase the risk
of introducing pests and pathogens into non-native landscapes \cite{POTTER201761, lovett2016nonnative, roy2014increasing}.
Epidemics caused by non-native pathogens can be catastrophic to tree populations that lack immunity and genetic
resistance to the invasive species \cite{doi:10.1002/9781444329988.ch8}; this can be understood from an evolutionary perspective: 
in an environment unaltered by human transportation, tree and plant species are thought to co-evolve alongside pests and pathogens
in a gene-for-gene like arm-race \cite{flor1971current, dangl2001plant, Thrall1735}. 

However, the introduction of a foreign
pathogen can overwhelm a population that has no such immunity \cite{desprez2016evolutionary}. Two classic examples that shook
the world are: Dutch elm disease \cite{doi:10.1111/j.1365-3059.2010.02391.x} in the United Kingdom and chestnut blight
\cite{doi:10.1002/9780470535486.ch7} in North America.

% Control of tree disease
Recently, the importance of effective trade regulations for preventative epidemic control has become apparent
\cite{rodoni2009role}. The role of shipping and human transport is an essential factor that risks the introduction
of invasive pests and pathogens into vulnerable landscapes within a country. For example, the shipping of elm timber
infected with scolytid bark beetles, carrying the fungus \textit{Ophiostoma novo‐ulmi} was identified as an essential
factor in driving the Dutch elm outbreak in the United Kingdom \cite{doi:10.1111/j.1365-3059.2010.02391.x}. 

Thus, adequate border controls are an indispensable step in a nations arsenal to stop the spread of disease.
Ordinarily, preventative measures take the form of customs checks on imported and exported plant material such
as timber, crops or horticultural goods. In particular, the European Commission enacted plant passports to regulate
how growers and traders can transport plant material between countries\footnote{In light of Brexit, the UK now plans
to implement an equivalent passport} \cite{wulfert2010implementation}.

If checks and policy implementations fail, a pathogen might be introduced into the landscape and spread through natural
dispersal pathways. Alternatively, a pathogen might be transported into foreign ecosystems through atmospheric 
long distance dispersal (LDD) \cite{brown2002aerial}. In either scenario, biological control becomes necessary. 
The biological control of plant-based disease
can be achieved in numerous ways. Commonly, this involves chemical agents such as pesticides, predatory insects or planting
genetically resistant cultivars \cite{pal2006biological, baker1974biological}. 

In this thesis, we are motivated to investigate the eradication of tree-based pathogens 
where eradication generally entails the removal of infected and diseased trees through sanitation felling \cite{pietzsch2021effect}. 
In this case, the questions to be answered are A) How do we effectively identify an infected tree? B) 
Which infected trees are the best choices to fell? C) What is the risk that a large-scale epidemic will result?


\section{Modelling and policy}

% \cite{thompson2016management}  - problems associated with 'control'
% \cite{gaydos2019forecasting} -  workshop with policy makers
% \cite{jones2020modelling} - developing plant health models in conjunction with decision makers 
% \cite{tsouvalis2019post} - ash dieback politics

The benefit of controlling an epidemic should outweigh the costs of letting an outbreak spread unchecked. 
Plant disease modellers can help infer well-designed control policies that maximally reduce epidemic impact and minimise the expenditure
of resources\textemdash both natural and economic. However, achieving this in practice is problematic due to various unknowns \cite{13-challenges}.
Moreover, history gives examples of insufficient control policies which fail to halt pathogen spread. 
Prominent examples include Dutch elm disease in the late 1960s and early 1970s \cite{dutch-elm-mismanage}, 
and more recently citrus canker in Florida \cite{schubert2001meeting}.

With mathematical models we can attempt to understand what dictates the optimal control of tree diseases epidemics. 
Control strategies have been explored on both smaller \cite{risk-potential-control} 
and larger landscapes \cite{large-scale-control2}. Current consensus on all spatial scales
agree that the scale of response must equal the scale of epidemic \cite{control-scale-matching}. In addition, 
any on-the-ground response must be carried out swiftly, otherwise the likelihood of successful management rapidly decreases and the cost of inaction soars.

Conventional eradication strategies involve detecting symptomatic trees and culling neighbours within a radius \cite{WEBIDEMICS}.
This is made difficult by numerous factors including the cryptic nature of tree diseases and resource constraints which may vary
over time \cite{control-theory, control-theory-application}. The naive strategy can be fine-tuned and optimised in many ways to increase
efficiency. One scheme involves ranking targets according to the risk they pose in order to prioritise culling \cite{risk-potential-control}.
Over large-scales, evidence suggests epidemics are most effectively controlled by targeting infected trees either at or ahead of
the infectious wave-front, as has been shown for sudden oak death in California \cite{large-scale-control}.


Optimal eradication complements enhanced `surveillance' and monitoring strategies, 
which also seek to optimise resource allocations. Here, the surveillance aims to detect infected individuals and infer disease incidence;
generally, this requires the collection and analysis of data on botanical epidemics \cite{surveillance-review}.
Ultimately, surveillance and monitoring comprise the last line of defence after preemptive boarder 
checks and inspections have failed to prevent the introduction of disease. 

Statistical approaches have been adopted to optimise the number of samples/surveys required to infer disease incidence accurately \cite{yamamura2016sampling}.
Moreover, optimal surveillance strategies have been examined with logistic \cite{parnell2012estimating} 
and mechanistic \cite{WEBIDEMICS} epidemic models\textemdash the next Chapter offers a more in depth review on these modelling paradigms.

\begin{figure}
    \centering
    \includegraphics[scale=0.35]{chapter1/figures/modelling-and-policy.pdf}
    \caption{A simplified model representing the major socioeconomic interactions between the general public, scientists, 
    policymakers and stakeholders in the UK. Scientists receive funding from and collaborate with Governmental bodies/policymakers. 
    Policymakers allocate resources and lead control initiatives to protect tree health. 
    Affected stakeholders can then join control initiatives and offer an on-the-ground response to pests and pathogens.}
    \label{fig:modelling-and-policies}
\end{figure}

\newpage

Even supposing an accurate and well-informed control strategy,
on-the-ground responses arise when stakeholders adopt policies or join control initiatives \cite{reed2018theory}.
Figure \ref{fig:modelling-and-policies} presents a simplified view of the interactions which dictate research output, public awareness, 
policy-making and the eventual epidemic control in the UK\footnote{
Figure \ref{fig:modelling-and-policies} was informed through informal conversations
through time spent at DEFRA/Fera (San hutton, York).}. Scientists in several disciplines\textemdash
from molecular biology to mathematics\textemdash receive funding from and collaborate with governmental
bodies, e.g. the Department for Environment Food \& Rural Affairs (DEFRA). Government-led control initiatives, resource
allocation and recommendations can then direct on-the-ground stakeholder\footnote{Here, `stakeholder' is a broad
term that reflects any interested individual, collective, or organisation 
(with a stake) that have the potential to influence a policy direction or control decision
\cite{brugha2000stakeholder}; this contrasts with the `public' that is not directly affected by control policies and have no vested interest in the policy direction.} 
responses. 

In addition, scientists can engage stakeholders directly (discussed more below) or influence public opinion through outreach and
scientific communication. In turn, the public can influence the decisions of policymakers through
mounting sufficient political pressure \cite{fuller2016public}.

Unfortunately, several blockers inhibit a well-informed, timely and effective response. 
In particular, poor accessibility to scientific research is widely-known to inhibit policy adoption, 
as disseminating scientific information requires in-depth domain knowledge and technical skill \cite{jones2020modelling}.
In a bid to make their work more accessible to policymakers and stakeholders, modellers have endeavoured to construct user-friendly interfaces\footnote{
The reader can find the user-friendly modelling interface constructed by \cite{WEBIDEMICS} at \nolinkurl{http://www.webidemics.com}} \cite{WEBIDEMICS}.
Other strategies to leverage scientific output involve directly facilitating discourse between modellers and stakeholders, categorised as `participatory modelling' (PM).

Recently, PM has become popular with risk and disaster management modelling research \cite{hamalainen2020leadership, ravera2020participatory, hedelin2017participatory}.
Nonetheless, PM approaches are rare in the context of plant disease, as reviewed by \cite{gaydos2019forecasting}.
In addition to a literature review, \cite{gaydos2019forecasting} held an interactive workshop with stakeholders
regarding the spread of \textit{P. ramorum} in the United States. The workshop facilitated stakeholder engagement
with an epidemic model \cite{tonini2017tangible} (reviewed in the preceding Chapter). In particular, the authors reported
that the stakeholders engaged well with the model and confirmed that the results were consistent with observations in the field. However, and most interestingly, most of the stakeholders with expert knowledge of the landscape remained sceptical of the host distributions accuracy. 
Such insights are hard to deduce for modellers who generally remain less connected to the actual landscape. 
As such, \cite{tonini2017tangible} demonstrated a solid and positive motivation to facilitate the collaboration 
between plant-health modellers and stakeholders through PM.

An effective response generally relies on widespread adoption of policies among multiple 
stakeholders, which is thought to depend on several additional factors. As an example, \cite{milne2020makes} coupled
an epidemic model of citrus huanglongbing disease (HLB) and stakeholder opinion dynamics. In the behaviour model, stakeholder
opinions depended on research, other citrus growers, consultants, and the media. The perception of risks and trust in
area-wide control led the stakeholders to join an area-wide control initiative. Subsequently, the analysis of 
\cite{milne2020makes} suggests that control efficacy facilitated stakeholder-engagement more than the perceived risks,
and that frequent contact between stakeholders and advisors increases the probability of successful control.



\section{Chapter summary}

In this thesis, our motivation is to develop robust epidemic models of infectious tree disease
epidemics from first principles. Then, more realistic and elaborate models can be constructed to
categorise epidemic severity over GB to help guide policymakers. 
In particular, previous large-scale investigations
have focused on specific pathosystems in a dynamic \textit{metapopulation-like} setting
\cite{large-scale-control, meentemeyer2011epidemiological, harwood2009epidemiological}. We take
an alternative approach and develop a general-purpose framework to spatially scale a small-scale epidemic
model (between individual trees) over large areas. The result is an $R_0$-map across GB with closer parallels
to the emerging field of Infectious Disease Cartography \cite{otieno2021modeling, KRAEMER201619, messina2016mapping}.

Chapter \ref{chapter2:litreview} begins by outlining several requisite modelling themes. First, the review recounts
several seminal works that founded the field of quantitative botanical epidemiology. Following this,
a suite of small, large and multi-scale spatial epidemic models are reviewed. Additionally,
the Chapter provides an account of host distribution datasets available in GB. Lastly, a case study of the emerging
ash dieback epidemic is presented.

Chapter \ref{chapter:SLM} sets the scene with a percolation-based simple lattice model of tree
disease spreading through a highly dense forest \cite{OROZCOFUENTES201912}. The model is compartmentalised
into an $SIR$ framework and demonstrates a sharp transition threshold above which an epidemic will propagate. 
Above the threshold of transition, a travelling wave-like behaviour is demonstrated.

Chapter \ref{chapter:SLM-applications} builds on the percolation-based model constructed in Chapter \ref{chapter:SLM}.
Firstly, we extend the work of \cite{OROZCOFUENTES201912} and present an alternative method to detect an early
warning signal in two-dimensional parameter space. Lastly, the epidemic model is coupled to a map of predicted
oak abundance over GB \cite{hill.data} to outline a `toy' large-scale model of tree disease. 
Primarily, Chapter \ref{chapter:SLM-applications} demonstrates that nearest-neighbour interactions are problematic
for realistic tree densities, which motivates an improved dispersal model. 

Chapter \ref{ch5:dispersal-model} introduces a generic (non-local) Gaussian dispersal kernel into the epidemic model. 
Tree disease epidemics are demonstrated to spread at lower, more realistic tree densities across GB and overcome the inherent nearest-neighbour limitations witnessed in Chapter \ref{chapter:SLM-applications}. Disease spread is then examined over a range of dispersal scale parameters and compared to the standard $SIR$ model.
Next, a spatially-explicit analytic expression for $R_0$ is derived and compared to a `\textit{contact tracing}' method of calculating $R_0$.
Both methods of determining the reproductive ratio are shown to demonstrate a threshold at $R_0=1$.

Chapter \ref{ch:6-adb} develops the generic (stochastic) dispersal model towards a simplified mechanistic
model reflecting the life cycle of ash dieback. The model involves four compartments, $SEIR$, that repeat
annually according to the sexual reproduction of ash dieback. Consequently, a method is presented to compose $R_0$-maps
across GB using the map of predicted ash abundance given by \cite{hill.data}. Lastly, a connected-component-analysis
(CCA) algorithm is used to visualise highly clustered regions of $R_0$-values. Examining $R_0$-map clustering as a function
of infectivity reveals behaviour akin to a global epidemic phase transition. That is, below a certain infectivity threshold, 
the pathogen would not be able to invade GB.

Chapter \ref{ch7:landscape-level-control} proceeds from observations discussed in \ref{ch:6-adb}. 
Namely, Chapter \ref{ch7:landscape-level-control} presents the first steps toward a novel landscape-level
control strategy based on the large-scale host structure. More specifically, the epidemic control strategy targets
natural pinch-points and fault lines in the spatial distribution of hosts, which, in theory, could bottleneck the epidemic
spread between at-risk regions. The Chapter ends by discussing the major assumptions in the control method and presents
a series of research questions that need to be addressed before the control method is demonstrated sufficiently.

Chapter 8 discusses the limitations and future developments of the work presented in this thesis.