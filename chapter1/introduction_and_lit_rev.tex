\chapter{Introduction and Literature Review}

This study will outline how to construct models of infectious tree disease for the purpose of epidemic control. Starting from a simple, parsimonious mathematical model of tree disease and limited information on parameter-values, I will show how to construct more elaborate models. As we move from simple model definitions, based on a uniform lattice, this thesis rigorously accesses limiting assumption in the model at each step as we move towards a more elaborate and sophisticated model. From a well constructed model, optimal large-scale epidemiological control will be investigated in an attempt to help inform policy makers about the best decision of resource expenditure and control efficiency. Throughout this study, I hope to provide useful techniques to capture and quantify the spread of disease through a population of trees.\\

This introductory chapter serves to inform the reader about the background, societal importance and modelling paradigms used in tree-based epidemiology. Firstly, the background and motivation for understanding tree disease will be reviewed. After which, a historical perspective will be outlined and I will conduct a through review into the various types of approaches that can be used to model the spread of disease through a population of trees. To this aim, several key concepts will be reviewed, including the basic reproduction number for plant-based diseases, the role of scale in transmission and dispersal. \newpage

\section{Background and Motivation}

In this study, I will focus on the mathematical and computational modelling tree-based epidemics in Great Britain. The problem of tree disease is vast and multifaceted, in reality, a country is likely to be effected by multiple threats involving diverse species of host and pathogen. Managing any a single outbreak is in itself an ambitious task, let alone managing simultaneous outbreaks on several fronts. However, we proceed.\\

\subsection{Trees-health and ecosystem function}
\begin{itemize}
    \item \textcolor{red}{I could expand on the economics, along with climatic and ecological importance of trees}
\end{itemize}
Trees grow naturally, in both rural and urban settings, this includes commercially managed plantations and orchards. Protecting and ensuring tree-health is of essential importance for society \cite{Boyd1235773}. Trees play a central role in terrestrial ecosystems \cite{boyd2013consequence} and their importance of forestry health is widely recognised. The motivations to protect tree populations in rural and urban landscapes are numerous. The most widely known, and well researched, themes include economic, climatic and ecological function \cite{freer2017tree}. However, the benefits of maintaining tree-health also stretch into of realms of human well-being, mental health, recreation and the perception of natural beauty \cite{tyrvainen2005benefits}. See Table \ref{table:tree-health} for a brief overview of the different benefits tree-health can offer to ecosystem and ecosystem function.\\

\subsection{The problem of tree diseases}
\begin{itemize}
    \item \textcolor{red}{I could expand on the biology of pests and pathogens and their interactions with trees}
\end{itemize}
% pest, pathogens and how they effect tree-health
Standing in the way of tree-health and ecosystem-function is the spread of disease through pest and pathogen vectors. Tree populations are subject to infection, the spread of disease and death just like human populations \cite{hethcote2000mathematics}. The main drivers of tree-based epidemics consist of fungi, bacteria, virus, oomycetes and instects \cite{manion1981tree}. To model the spread of tree disease, we require a sound knowledge of the general biological interactions involved. Thus, it is necessary to understand some distinctions between pests and pathogens and introduce some important examples and nomenclature.\\

The terms `pest' and `pathogen' are both broad referring to taxonomically diverse organisms. A pest is defined as any organism that harms humans or human interests such as crops or livestock \cite{oerke2006crop, de1964biological, buckle2015rodent}. The main pest-threats to tree species are typically insects \cite{metcalf1994introduction}. As an example, the Asian longhorn beetle (ALB) \cite{haack2010managing} and oak processionary moth OPM \cite{tomlinson2015managing} are two pests that currently threaten trees in Great Britain. On the other hand, an organism is a pathogen if it causes disease \cite{balloux2017q}. In the context of tree disease this includes fungi, bacteria, virus, oomycetes \cite{Boyd1235773}. In Great Britain, \textit{Phytophthora ramorum} \cite{brasier2005phytophthora} and Ash dieback \textit{Hymenoscyphus fraxineus} \cite{mitchell2014ash, ash-dieback-costs} are two pathogens that threaten tree-health. Some of the most pressing threats to Great British trees are provided in Table \ref{table:tree_threats}.\\

\subsection{The problem of human trade and transport}
\begin{itemize}
    \item \textcolor{red}{Expand on the plant-passport, how do I cite government reports ?}
\end{itemize}
% why globalised modern world is more at risk than ever
It is widely accepted that trade and transport of foreign plant material, through imports and exports, has increased the risk of introducing pests and pathogens into non-native landscapes \cite{POTTER201761, lovett2016nonnative, roy2014increasing}. Epidemics caused by non-native pathogens can be catastrophic to tree populations which may lack immunity and genetic resistance to the invasive species \cite{doi:10.1002/9781444329988.ch8}. This can be understood from an evolutionary perspective. In an environment unaltered by human transportation, tree and plant species are thought to co-evolve alongside pests and pathogens in a gene-for-gene like arm-race \cite{flor1971current, dangl2001plant, Thrall1735}. However, the introduction of a foreign pathogen can overwhelm a population which has no such immunity \cite{desprez2016evolutionary}. Two classic examples that shook the world are: Dutch elm disease \cite{doi:10.1111/j.1365-3059.2010.02391.x} in the United Kingdom and chestnut blight \cite{doi:10.1002/9780470535486.ch7} in North America. Thus, given our dependence on globalised trade, the threat to trees, flowering plants and crops grows evermore alarming.\\ 

% Control of tree disease
In recent years the importance of effective trade regulations, for the purpose of preventative epidemic control, has become apparent \cite{rodoni2009role}. The role of shipping and human-transport is an important factor which risks the introduction of invasive pests and pathogens into vulnerable landscapes within a country. As a case in point, the shipping of elm timber infected with scolytid bark beetles, carrying the fungus \textit{Ophiostoma novo‐ulmi}, has been identified as and important epidemic driver in the Dutch elm outbreak in the United Kingdom \cite{doi:10.1111/j.1365-3059.2010.02391.x}. Effective boarder controls are an important step in a nations arsenal to stop the spread of disease. Ordinarily, these preventative measures take the form of custom checks on imported and exported plant material such as timber, crops or horticultural goods. The common use of plant passports\footnote{Enacted by European Commission in 2017, \textcolor{red}{Not sure how to cite government report.}} is a an example of a large-scale policy regulating checks on the trade and transport of plant goods. 

\subsection{The natural spread of plant-based diseases}
If checks and policy implementations fail, a pathogen might then be introduced into the landscape and start spreading through natural pathways via dispersal. At this point, the biological control may become a necessary. The biological control of plant-based disease can be achieved in numerous ways, commonly this can include chemical agents such as pesticide, predatory insects or planting genetically resistant cultivars \cite{pal2006biological, baker1974biological}. In this thesis, we are motivated investigate the eradication of tree-based pathogens. In this scenario control, we are typically limited to eradicating infected and diseased trees through sanitation felling. The questions to be answered in this case are: A) How do we effectively identify an infected tree? B) Which infected trees are the best choices to fell ? C) What is the risk that a large-scale epidemic will result ?\\

\subsection{Epidemic control of plant-based diseases}
\begin{itemize}
    \item \textcolor{red}{Maybe more from a government prospective ? Include some arrow diagrams for the decision chain and how it works in the UK.}
\end{itemize}
% control and the benefits
The benefit of controlling an epidemic should outweigh the costs of letting an outbreak spread unchecked. A well designed control policy should maximally reduce epidemic impact and minimise the expenditure of resources\textemdash both natural and economic. Achieving the optimal control of an epidemic in practice is a challenge due to various unknowns \cite{13-challenges} and history gives examples of insufficient control policies that failed to halt pathogen spread. The management and policy implementations of citrus canker in Florida \cite{schubert2001meeting} and Dutch Elm disease in Great Britain \cite{dutch-elm-mismanage} serve as two stark reminders. In both of these these scenarios, policy makers were slow to act and did not sufficiently comprehend the scale of the problem before it was too late. So, efficient control relies on well informed strategies. Moreover, understanding gained through accurate modelling and informed policy go hand-in-hand \cite{jones2020modelling}.\\

% Mathematical modelling as an antidote
With mathematical models, we can attempt to understand what dictates optimal control of tree diseases. Strategies have been explored on small-scale \cite{risk-potential-control, WEBIDEMICS} and large-scale landscapes \cite{large-scale-control, large-scale-control2}. Currently, consensus on all spatial scales agree that the proportionate response must equal the scale of epidemic \cite{control-scale-matching}. Furthermore, any response must be carried out swiftly, otherwise the likelihood of successful management decreases rapidly and the cost of inaction soars.\\

\begin{table}
    \centering
    \begin{tabular}{|p{3cm}||p{13cm}| }
    \hline
    \textbf{Benefit}&\multicolumn{1}{c|}{}\\
    \hline
    Social  & Recreational activities, mental and physical health, cultural and historic sentiment.\\
    \hline
     Aesthetic & Landscape variation, textures, colours. Seasonal dynamics which change landscape views. \\
    \hline
     Climatic &  Cooling, wind control, impacts on urban climate through temperature and humidity control. Air pollution reduction, sound control, glare and reflection on reduction, flood prevention and erosion control. \\
    \hline
     Ecological & Biodiversity and biotopes for flora and fauna.\\
    \hline
     Economic & Timber, wood pulp, fiber and food.\\
    \hline
    \end{tabular}
    \caption{The benefits of tree health, based on \cite{tyrvainen2005benefits} and \cite{boyd2013consequence}}
    \label{table:tree-health}
\end{table}

\begin{table}
    \begin{tabular}{ |p{3cm}||p{3cm}|p{3cm}|p{6cm}|  }
     \hline
     \multicolumn{4}{|c|}{Pest and Pathogens} \\
     \hline
     \textbf{Classification} & \textbf{Name} & \textbf{Host} & \textbf{Symptoms} \\
     \hline
     Ascomycete fungus & Ash dieback\newline \textit{Hymenoscyphus fraxineus} & Ash tree\newline \textemdash{Fraxinus} & Blacking wilt on leaves and leaf loss. Eventual cankers of branches and trunk \\
     \hline
    \end{tabular}
    \caption{A selection of the most threatening pests and pathogens to tree health in Great Britain.}
    \label{table:tree_threats}
\end{table}

\section{A Historical Perspective: Modelling Tree Disease}

The theory of human epidemics has a rich history routed in superstition. History records the first objective attempt in by Hippocrates (460 \textemdash 370 BC) \cite{langholf2011medical}. The first quantitative mathematical model came from \cite{kermack-model}, the so called $SIR$ model. On the other hand, plant pathologists wishing to quantify the growth of plant and crop-based epidemics had no mathematical framework and had wait until Van der Plank published his seminal \cite{van2013plant}. Van der plank used a logistic growth model, predicated on the growth of money, to capture the essential population dynamics of healthy and infected plants.\\

\subsection{Plant-pathology meets epidemiology}
Historically, plant pathology had no cross-fertilization with epidemiology. The two fields existed in two very different spheres with early plant pathologists tending to focus on microscopic details of pathogen growth in a biological setting. Initially, the logistic-growth formulation found application most notably in the management of crops \cite{browning1969multiline}. The logistic growth model presented by Van der Plank can be seen as simple one-compartment variant of the original $SIR$ framework developed by \cite{kermack-model} and marked the transition into a more quantitative discipline.\\

Over the ensuing few decades, several others key worker, such as \cite{zadoks1979epidemiology} and \cite{campbell1990introduction}, consolidated the now well-defined field of plant-based epidemiology. 
After the initial coalescence of plant pathology and mathematical epidemiology, epidemics in urban and rural tree populations were naturally ripe to be explored with the same tool sets \cite{manion1981tree}. \textcolor{red}{first epidemiological analysis of dutch elm...}.

\subsection{The age of simulation}

Advances in plant-based epidemiology around the initial period 1960-1980 was compounded by developments in computing. Improved commercial availability and computer architectures allowed for faster calculations and permitted the use of more intensive models. The first epidemic simulator, named `EPIDEM', written in FORTRAN IV came by \cite{waggoner1969epidem}. This early simulator modelled the spread of the fungal `Alternaria solani', that effects potatoes and tomatoes. The growth of infected tissue within leaves was considered under the influence of different abiotic conditions. A variety of computational models were put forward \cite{doi:10.1146/annurev.py.23.090185.002031} and the ability to simulate more intricate models grew in proportion the amount of computer memory available \cite{zadoks1972methodology}. With more computational power, more information about the host-pathogen interaction can be simulated in a reasonable time.\\

Modelling the spread of disease in tree populations is subject to the same approach as modelling crop-based disease\footnote{Sometimes they are one and the same, \textit{Xyella Fastidiosa} for example is a pathogen that effects crops in the olive tree \cite{doi:10.1146/annurev-phyto-080417-045849} among others \cite{simpson2000genome}.}. However, there are some notable difference reflected in the models, \textcolor{red}{find examples, time-scales, sporulation rates, scale and outcome e.g. crops food-security}.

\section{Mathematical modelling approaches}

\subsection{Determinism and stochasticity}
\textcolor{red}{
\begin{itemize}
    \item Explain differences in modelling approaches
    \item Why use stochastic methods
    \item How do you collect results of a stochastic model ? Ensemble averaging
    \item Give an overview of some results in the literature
\end{itemize}}

\subsection{Lattice-based modelling}
\textcolor{red}{
\begin{itemize}
    \item Agent-based modelling
    \item A good place to talk about Percolation models
\end{itemize}}
\blindtext

\subsection{Dispersal}
\textcolor{red}{
\begin{itemize}
    \item The importance of dispersal
    \item How is dispersal treated mathematically ?
    \item What type of dispersal kernels have been studied?
    \item What parameter-values have been inferred ?
    \item How long-range can dispersal be ? Talk about long-range inter-Continental dispersal
\end{itemize}}
\blindtext

\subsection{Network Modelling}
\textcolor{red}{
\begin{itemize}
    \item What paradigm do network models suit?
    \item Human trade and transport and networks 
    \item plant-to-plant interactions and networks ? 
    \item give examples of how network models have been used in the literature
    \item review \cite{doi:10.1098/rsif.2005.0051}
\end{itemize}}
\blindtext

\section{Important concepts: plant-based epidemics}

\subsection{The role of scale}
\textcolor{red}{
\begin{itemize}
    \item The spatial aggregation of vegetation, how does clustering look on different spatial resolutions ?
    \item Spatial scale in dispersal ? What does this mean for disease gradients ?
\end{itemize}}
\blindtext

% Consider doing a chapter on `thresholds` that can incorporate Invasion & Persistence
\subsection{Thresholds in plant-based disease}
\textcolor{red}{
\begin{itemize}
    \item What was the first paper to use this term ? Give mini-historical recount
    \item What is Invasion and persistence ? Give examples of papers and results in literature 
    \item The importance of thresholds
    \item Where does this number come from and why is \textbf{}this an important number ?
    \item How can one define $R_0$ ? Talk about how the there are different ways to calculate it e.g. next-generation operator 
    \item Survey what results have been obtained in the literature using $R_0$
\end{itemize}}

\subsection{The basic reproduction number}
The basic reproduction number, denoted by $R_0$, can be seen to arise in the study of population dynamics and demographics to categorize offspring \cite{heesterbeek2002brief}. In the study of epidemics, $R_0$ was adapted in order to categorize the number of infectious offspring and is widely agreed to be the most important and informative parameter. The basic reproduction number can be derived from the $SIR$ model \cite{kermack-model}, and from it, we can see how a likely disease is to spread though a population. Numerous definitions of $R_0$, and methods of determination, have been proposed in the literature resulting in confusion and multiple $R_0$ values for one pathogen \cite{delamater2019complexity}. However, a common definition states:

\textit{The expected number of secondary infections that result from one infected individual, over it's entire infectious lifetime, in a completely susceptible population.}

here, the infectious life-time is the total amount of time an individual remains infectious. The individuals infectious status will culminate in either recovery or death. Importantly, from $R_0$ a threshold can be determined which dictates whether or not the disease will spread through large parts of the population. That is, if $R_0>1$ an \textit{epidemic} will result, otherwise the spread of disease will likely halt. The value of $R_0$ can be determined from different methods\footnote{Common approaches include, the survival function, next-generation operator and estimation from epidemiological data \cite{perspectives-on-r0}} and from it many surrogate parameters can be calculated which leads to confusion \cite{diekmann2010construction}.\\

The basic reproduction number is used extensively by researchers wishing to understand the spread of disease through human populations. In the study of plant-based disease however, the basic reproduction number has attracted far less attention despite being a convenient parameter from which a threshold can be derived. Thresholds are of great importance to understand epidemics in plant-based populations. A threshold criteria similar to the $R_0$-threshold was first introduced by \cite{van2013plant} in the logistic growth model. However, the logistic-growth based threshold was a vast simplification and had limitations in accurately predicting the threshold for growth and spread \cite{onstad1992evaluation}.\\

The basic reproduction number $R_0$ was studied by \cite{gubbins2000population} in order to understand thresholds in plant-parasite \textit{pathosystems}. A plant-host under attack from a parasite may respond to the \textit{parasite load} with either the promotion or inhibition of susceptible tissue, see \cite{gilligan1997analysis} for further details on this mechanism. In order to model this problem, \cite{gubbins2000population} used a system of three linked differential equations describing the density of susceptible hosts, infected hosts and the primary source of innoculum\footnote{\textemdash here, innoculum is a general term that refers to any part of the parasite that, when in contact with the plant, may induce disease \cite{agrios2005chapter}}. Transmission of innoculum could occur through a dual source, that is, through primary or secondary pathways. Local stability analysis on the linked equations lead to various $R_0$ values, and subsequently threshold criteria, dependant on the functional form assumed by the host-response.\\


\subsection{Invasion and Persistence}

When calculating thresholds, the spatial structure of the host population cannot be overlooked. A study conducted by \cite{park2001invasion} considered how spatial heterogeneity effects the dynamics of plant-parasite interactions and derived the basic reproduction number for a spatially structured host population. The density of susceptible and infected hosts were comparatively examined with deterministic and stochastic model variants. The host population in \cite{park2001invasion} consisted of \textit{patches}, which in reality could be either a single host, field or region of land. Inside each patch, the \textit{within-patch} dynamics were governed by a basic reproduction number, $R_p$. If $R_p > 1$ the parasite could survive and reproduce locally or otherwise die. The infection could jump between different patches via a longer-range interaction, set by a coupling strength. Patches were locally coupled together in a neighbourhood and the infection could jump between neighboring patches. The strength of interactions between neighboring patches were set by a coupling strength $\epsilon$ which was independently of distance.\\

From the model put forward by \cite{park2001invasion}, the concept of \textit{persistence} could be observed. Persistence occurs when a pathogen can reproduce, and thus spread from host-to-host, however the the spread is slow enough such that host-regrowth can keep continually supplying the pathogen with susceptible matter. Whereas, if the pathogen spread more aggressively through the population, the supply of susceptible material would be quickly exhausted and the pathogen would quickly die.\\

% $R_0$ is a complex function which changes in time, to this end, the next generation operator is used to derive a value for $R_0$ \cite{doi:10.1098/rsif.2009.0386}.
% ref 2 \cite{doi:10.1146/annurev.phyto.011108.135838} ref 3. \cite{van2011periodic}
