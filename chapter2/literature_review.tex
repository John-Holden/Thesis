% -----------------------------------------------------------------------------
% 1) Paint a story line of developments beginning from SIR
% 2) Give adequate breakdown of types of models:
%.   -  Percolation-based lattice models 
%    -  Non-local dispersal models culminating in the meta-population model
%    -  The continuum-based PDE models ? Spatial SIR models
%.   -  Network models ? 
% 3) Give a breakdown of control methods with these types of models
%.   - Clear examples of how mathematical and computational models inform policy
% -----------------------------------------------------------------------------


\chapter{Tree diseases epidemics: an interdisciplinary problem (WORK IN PROGRESS)}
\label{chapter2:litreview} 

The mathematical and computational modelling of infectious tree diseases began to emerge after Plank had others put forward seminal works on the modelling of crops. Tree disease is import to consider for tree-health in both commercially cultured of stands and naturally occurring forestry and ecosystems. The modelling tree-based disease has many features in common with modelling crop diseases. However, differences in the spatial distribution of hosts and growth cycles remain large.

The aim of tree disease modelling is to help design effective control policies. From well designed control policies, the spread of disease through commercial and rural environments can be managed and tree-health maintained. This chapter begins with a chronological overview of tree disease models and the different approaches that have been used in the past.

Having reviewed typical models in literature, the theme of control will be reviewed. That is, how the optimal control of a pathogen can be investigated from the construction of models.

Modern Epidemiology is extremely interdisciplinary utilising various tools from mathematics, physics and computer science. The project will explore infectious diseases specific to tree species. There have been various models used to model tree disease including: 1) Lattice-based percolation. 2) Continuum. 3) Meta-population. 4) Network models. A summary of these models will be given in this chapter along with their main usage, control. A review of epidemic control in plant populations will be given in support of chapter where we detail a novel method stop the spread of disease. 


\section{Determinism and stochasticity}

The spread of disease inherently takes the form of a stochastic process, whereby an epidemic unexpectedly arises and quickly cease, seemingly out of the blue. In order to capture the spread of disease, stochastic effects cannot be overlooked... 

% \textcolor{red}{
% \begin{itemize}
%     \item Explain differences in modelling approaches
%     \item Why use stochastic methods
%     \item How do you collect results of a stochastic model ? Ensemble averaging
%     \item Give an overview of some results in the literature
% \end{itemize}}


\section{Compartmentalised models}
\label{ch2:lit-rev-compartmentalised-models}
Compartmentalised models are the dominate theme in epidemiological modelling...

% 1 - early work that decomposed KMc model 
\cite{segarra2001epidemic}

% 2 - describe how plank model can be represented in terms of a compartmentalised model
The authors of \cite{time-varying-infectivity} demonstrated that different sporulation functions can be derived from a more general class of $SEmInR$ models. 
Sporulation functions can then be used to describe lifetime of transitions between $E\rightarrow I$ along with the time-dependency of infectivity rates in the transition $S\rightarrow E$. 
In the seasonal model of ADB, $\phi_1(t)$ and $\phi_2(t)$ control the time-dependency of $\beta$ and the transition $S\rightarrow E$ alone.
The transition of ash between states $E\rightarrow I$ occurs in the first autumn leaf-shedding after infection\footnote{In theory, the transition of ash between states $E\rightarrow I$, set by a normal distribution centred in autumn, could be represented by taking a suitability large limit $m$ in the $SEmInR$ model proposed by \cite{time-varying-infectivity}.}. 

% In his seminal work, van der Plank \cite{van2013plant} considered a latency and infectious period, with a constant rate of infection while infectious; this model is represented by the equation:
% \begin{equation}
% \label{eq:plank-model}
%     \frac{dI(t)}{dt} = R\big(I(t - p)  - I(t - p - i)\big)\big(1 - I(t)\big)
% \end{equation}

% where $p$ and $i$ are constant latency and infectivity periods and $R$ is the corrected infection rate (i.e. including host demography). Equation \ref{eq:plank-model} is seldom used in contemporary literature, owing to the difficulty in analytically solving its time-delay \cite{madden2006botanical}. However, it provides a simple basis to understand time-varying infectivities. 

% Later, \cite{time-varying-infectivity} demonstrated that sporulation can be decomposed into a suite of $SEIR$-like models. Specifically, the authors of \cite{time-varying-infectivity} demonstrated that both compartmentalised $SEIR$ models and the plank formulation can be decomposed into a $SEmInR$ model i.e. where transitions follow $S\rightarrow E_1, E_2,.., E_m \rightarrow I_1, I_2, ..., I_n \rightarrow R$.
% By taking suitable limits of $n$ and $m$ the plank and $SEIR$ models can be recovered. 


%
\subsection{Lattice-based modelling}
Lattice-based models have been used to describe the spread of disease in many contexts, such as the spread of malaria, the bubonic plague and the spread of tree disease...

% \textcolor{red}{
% \begin{itemize}
%     \item Spatial SIR model
%     \item Agent-based modelling
%     \item A good place to talk about Percolation models
%     \item \textcolor{red}{Murray}: travelling waves that do not change shape
% \end{itemize}}

\section{The role of scale}
% WORK IN PROGRESS
The role of scale is crucial to describing the spread of tree-based disease...
\blindtext
% \begin{itemize}
%     \item The spatial aggregation of vegetation, how does clustering look on different spatial resolutions ?
%     \item Spatial scale in dispersal ? What does this mean for disease gradients ?
%     \item  \cite{https://doi.org/10.1111/jbi.13642}
%     \item  \cite{pautasso2013european}, search for 'modelling' in this review paper. It contains references to LDD models.
%     \item LDD is an important factor of the scale of spread. Long distance dispersal has been considered for many biological processes including seed and pollen dispersal \cite{baguette2012dispersal, levin2003ecology}.
%     \item LDD is highly stochastic \cite{nathan2006long}
%     \item \cite{wingen2013long} demonstrated how host heterogeneity and distance between patches can...
%     \item traditional model of plant disease consider a wave-like spread of disease, a biological wave-like phenomena, that spreads from region to region (CITATIONS...).
%     However, this is not the case, as demonstrated by \cite{mundt2009aerial}. Epidemic fronts and disease gradients can range over multiple scales. The spread of disease through plant-based populations is thus, a multi-scale problem.
%     \item indeed, fungal spores can disperse over large distances, as confirmed by modelling work \cite{shaw2006assembling} 
% \end{itemize}

\subsection{Dispersal}
One of the key features that distinguish tree disease from human-based disease is dispersal. The dispersal of a pathogen may take many forms..
\blindtext
% \begin{itemize}
%     \item The importance of dispersal
%     \item How is dispersal treated mathematically ?
%     \item What type of dispersal kernels have been studied?
%     \item What parameter-values have been inferred ?
%     \item How long-range can dispersal be ? Talk about long-range inter-Continental dispersal
%     \item see \cite{nathan2012dispersal} for a review of dispersal kernels
% \end{itemize}

\section{Modelling landscape-level epidemics}

Modernity has witnessed the spread of tree-based epidemics at the regional, country and continental level. 'landscape-level'

\subsection{The Metapopulation model}
\blindtext
% \begin{itemize}
%     \item this variation in epidemic outcome from variants of environmental factors was likened to the metapopulation used in ecology and population dynamics in order to describe environment `patchiness'
%     \item variability in the host landscape was considered and recognised as important, the patchiness of a landscape lends itself well to a metapopulation model
%     \cite{doi:10.1098/rstb.1986.0072}
%     \item \cite{large-scale-control}
% \end{itemize}

\subsection{Network Modelling}

Network models are useful to describe the spread disease of human-based trade and transport of infectious trees...

\blindtext

% \textcolor{red}{
% \begin{itemize}
%     \item What paradigm do network models suit?
%     \item Human trade and transport and networks 
%     \item plant-to-plant interactions and networks ? 
%     \item give examples of how network models have been used in the literature
%     \item review \cite{doi:10.1098/rsif.2005.0051}
% \end{itemize}}

% Consider doing a chapter on `thresholds` that can incorporate Invasion & Persistence
\section{Thresholds in plant-based disease}

A key theme in the spread of tree-disease, and disease in the broadest sense, can be understood through a threshold...
\blindtext
% \textcolor{red}{
% \begin{itemize}
%     \item What was the first paper to use this term ? Give mini-historical recount
%     \item What is Invasion and persistence ? Give examples of papers and results in literature 
%     \item The importance of thresholds
%     \item Where does this number come from and why is \textbf{}this an important number ?
%     \item How can one define $R_0$ ? Talk about how the there are different ways to calculate it e.g. next-generation operator 
%     \item Survey what results have been obtained in the literature using $R_0$
% \end{itemize}}

\subsection{The basic reproduction number}
The basic reproduction number, denoted by $R_0$, can be seen to arise in the study of population dynamics and demographics to categorize offspring \cite{heesterbeek2002brief}. In the study of epidemics, $R_0$ was adapted in order to categorize the number of infectious offspring and is widely agreed to be the most important and informative parameter. The basic reproduction number can be derived from the $SIR$ model \cite{kermack-model}, and from it, we can see how a likely disease is to spread though a population. Numerous definitions of $R_0$, and methods of determination, have been proposed in the literature resulting in confusion and multiple $R_0$ values for one pathogen \cite{delamater2019complexity}. However, a common definition states:

\textit{The expected number of secondary infections that result from one infected individual, over it's entire infectious lifetime, in a completely susceptible population.}

here, the infectious life-time is the total amount of time an individual remains infectious. The individuals infectious status will culminate in either recovery or death. Importantly, from $R_0$ a threshold can be determined which dictates whether or not the disease will spread through large parts of the population. That is, if $R_0>1$ an \textit{epidemic} will result, otherwise the spread of disease will likely halt. The value of $R_0$ can be determined from different methods\footnote{Common approaches include, the survival function, next-generation operator and estimation from epidemiological data \cite{perspectives-on-r0}} and from it many surrogate parameters can be calculated which leads to confusion \cite{diekmann2010construction}.

The basic reproduction number is used extensively by researchers wishing to understand the spread of disease through human populations. In the study of plant-based disease however, the basic reproduction number has attracted far less attention despite being a convenient parameter from which a threshold can be derived. Thresholds are of great importance to understand epidemics in plant-based populations. A threshold criteria similar to the $R_0$-threshold was first introduced by \cite{van2013plant} in the logistic growth model. However, the logistic-growth based threshold was a vast simplification and had limitations in accurately predicting the threshold for growth and spread \cite{onstad1992evaluation}.

The basic reproduction number $R_0$ was studied by \cite{gubbins2000population} in order to understand thresholds in plant-parasite \textit{pathosystems}. A plant-host under attack from a parasite may respond to the \textit{parasite load} with either the promotion or inhibition of susceptible tissue, see \cite{gilligan1997analysis} for further details on this mechanism. In order to model this problem, \cite{gubbins2000population} used a system of three linked differential equations describing the density of susceptible hosts, infected hosts and the primary source of innoculum\footnote{\textemdash here, innoculum is a general term that refers to any part of the parasite that, when in contact with the plant, may induce disease \cite{agrios2005chapter}}. Transmission of innoculum could occur through a dual source, that is, through primary or secondary pathways. Local stability analysis on the linked equations lead to various $R_0$ values, and subsequently threshold criteria, dependant on the functional form assumed by the host-response.\\

\subsection{\textcolor{red}{Invasion and Persistence}}
\label{ch3:invasions_and_persistence}
When calculating thresholds, the spatial structure of the host population cannot be overlooked. A study conducted by \cite{park2001invasion} considered how spatial heterogeneity effects the dynamics of plant-parasite interactions and derived the basic reproduction number for a spatially structured host population. The density of susceptible and infected hosts were comparatively examined with deterministic and stochastic model variants. The host population in \cite{park2001invasion} consisted of \textit{patches}, which in reality could be either a single host, field or region of land. Inside each patch, the \textit{within-patch} dynamics were governed by a basic reproduction number, $R_p$. If $R_p > 1$ the parasite could survive and reproduce locally or otherwise die. The infection could jump between different patches via a longer-range interaction, set by a coupling strength. Patches were locally coupled together in a neighbourhood and the infection could jump between neighboring patches. The strength of interactions between neighboring patches were set by a coupling strength $\epsilon$ which was independently of distance.

From the model put forward by \cite{park2001invasion}, the concept of \textit{persistence} could be observed. Persistence occurs when a pathogen can reproduce, and thus spread from host-to-host, however the the spread is slow enough such that host-regrowth can keep continually supplying the pathogen with susceptible matter. Whereas, if the pathogen spread more aggressively through the population, the supply of susceptible material would be quickly exhausted and the pathogen would quickly die.

% $R_0$ is a complex function which changes in time, to this end, the next generation operator is used to derive a value for $R_0$ \cite{doi:10.1098/rsif.2009.0386}.
% ref 2 \cite{doi:10.1146/annurev.phyto.011108.135838} ref 3. \cite{van2011periodic}

\section{\textcolor{red}{Controlling Plant-Based Epidemics}}

\blindtext

% Epidemiological models are commonly used to for 'control'. Spatial aggregation of the host landscape has been shown to play an important role in effective management. In their paper \cite{doi:10.1094/PHYTO-100-7-0638}
% showed that the optimal radius of control depends strongly on the level of aggregation (clustering).

% \begin{itemize}
%     \item Describe how control has been approached ?
%     \item What type of models have been used to access control ? \cite{WEBIDEMICS} \cite{large-scale-control}
%     \item what is the dominant narrative in controlling plant-based, and by extension tree-based, disease ? i.e. the scale of control must equal the scale of the outbreak.
% \end{itemize}

\section{\textcolor{red}{Host data sets: epidemiology and landscape ecology}}

% \textbf{This couples tree disease section to percolation and provides a good intro linking the above paragraph:}
% \begin{itemize}
%     \item A proper investigation of diseases spread through a tree population cannot be studied in isolation from the landscape. Disease spreads based on the spatial pattern of hosts, heterogeneity of resistance and `reservoir'? Landscape features play an important role in shaping the evolution of the spread. Forest pathology demands a landscape or spatial perspective. Broad scale epidemic (opposed to local outbreaks) require a landscape perspective, see \cite{pub.1012384986} for a review. \textbf{see \cite{pub.1073292723} for an early consideration of spatial considerations... ? double check}
%     \item \cite{doi:10.1098/rstb.1998.0226} gave a three population (host, parasite and hyperparasite), which considered transmission via a reaction-diffusion process.
%     \item including spatial components within an epidemiological frame-work demanded an interdisciplinary approach between forest pathologists and landscape ecologists, this combined knowledge of plant-based diseases with the methods and tools needed to present a working model in space.
%     \item remote sensing tools were used \cite{kelly2002monitoring} to track the study of sudden oak death in California
%     \item The need for spatial considerations within disease management was widely accepted at this point,
%     \item \cite{kelly2002landscape} demonstrated clustering at the landscape level, spatial structure is important
%     \item ... 
%     \item various frameworks were put forward, \cite[see][for a detailed analysis]{Gilligan-disease-management}
%     \item Percolation theory was used by bailey et al 2000, \textbf{find citations...}
%     \item the link between percolation and managing crops were obvious, 
%     \item dispersal has been modelled as a random-walk process \cite{PhysRevE.67.031913}
%     \item dispersal modelled in relation to heterogeneity?? \cite{CARACO2001185}
%     \item anthropogenic factors were noted \cite{doi:10.1890/0012-9658(2002)083[3167:SOAIPO]2.0.CO;2}, along with abiotic factors \cite{doi:10.1046/j.1439-0434.2003.00730.x} such as sunlight intensity ? double check
%     \item \cite{doi:10.1046/j.1442-9993.2002.01202.x} the spatial scale is important when modelling the process \cite{spatial-scale}
%     \item \textbf{introducing space into the problem lead to percolation...}
% \end{itemize}

Data regarding different tree species distributions will predominantly come in two forms, abundance and presence/absence data. Using abundance data is usually represented in $km^2$ tree cover per hectare (or percentage cover), whereas presence only data only gives binary information if a tree species is present or not.

Abundance data is preferred as more information is captured about the tree species, however, good quality abundance data is in short supply. An interesting model put forward by Louise et at \cite{2STAGE} uses a two stage distribution model. First, presence only data is combined with a series of environmental covariates using a species distribution model in order to produce a map of predicted occurrence data. Random forest regression is then applied to a training sample of real life abundance data producing a high resolution map predicting abundance over the UK at a resolution of $1km^2$. The final result was compared with the remaining abundance data and found to perform much better than existing models. Moving to realistic tree distributions is preferable as the parameter space is reduced by the removal of $\rho$.

% The accuracy of modeled canopy cover data depend on the quality, and quantity, of data sources. Key species that have greater ecological and economic importance, such as ash or oak, typically have better data availability.

\section{Percolation\textemdash from forest fires to epidemics}
\label{section:lit-rev-perc}

The original formulation of percolation theory was first used to describe properties of a fluid and the bonds which form between molecules \cite{perco_origin} (the reader may jump to \ref{fig:ch3-perc-invariance} for an account of percolation). The problem was posed on a graph\textemdash illustrated in terms of vertices and edges. However re-interpretations were subsequently put forward by physicists studying material sciences, naturally on a lattice. \cite{Essam_1980}. The attractive feature of this new paradigm was that percolation demonstrated a phase-transition. Thus, percolation could be treated with scaling theory used in the study of critical-phenomena. Accordingly, early work rigorously ensued to map out the behaviour of percolation around criticality in terms of critical exponents \cite{STAUFFER19791}. 

Different flavours of percolation models, such as, site or bond percolation were described to model different processes. Percolation proved a convenient theory and various phenomena including, gelation, magnetism and telecommunications were described \cite{trove.nla.gov.au/work/26493727}. Interestingly, forest fires were also applicable to a description within percolation theory \cite{MacKay_1984}. This was a short conceptual jump from time-dependant percolation used to study the growth of crystals \cite{Family_1985}. 

A fire spreading through a population of trees is not to dissimilar to a disease spreading through a population. This lead to a general epidemic-formalisation within percolation-based framework when \cite{pub.1059067807} argued that epidemics were in the same universality class as percolation. It is alluring to remark how modelling the same processes with different microscopic interactions (e.g. different lattice geometries) lead to the macroscopic behaviour should they have the same universality class\textemdash see chapter \ref{fig:ch3-perc-invariance} for more information.

Beginning with a simple $SIR$ framework put forward by \cite{kermack-model}, the field of epidemiology was already well established around the time percolation theory was conceived \cite{baily1975mathematical}. In particular, initial assumptions about population mixing could be relaxed with more elaborate spatial-contact models, naturally, percolation theory could serve as a useful tool when developing spatially-structured epidemic models.

A fractal-like pattern of epidemics was observed by \cite{GRASSBERGER1986273}, shown in Figure \ref{fig:1d_perc_basis}. Parsimoniously describing Figure \ref{fig:1d_perc_basis}, lighter grey sites represent removed individuals, black sites indicate actively infected sites and white sites indicate unaffected sites. All lattice sites in the bottom row were initially infected and the infection can be seen to propagate from the bottom up. The lattice was initialised at the critical-density $p\sim p_c$ culminating in a fractal-like pattern. \cite{GRASSBERGER1986273} did not attribute the hosts of this model to be trees, but instead a general host-population with very low mobility. Local interactions between hosts and infected were elucidated as vast simplifications and long-range interactions (following a power-law) were discussed\textemdash we shall see how these ideas form a important component of modelling tree disease.

\begin{figure}
    \centering
    \includegraphics{chapter2/figures/perc1.jpg}
    \caption{A space-time representation of an epidemic spreading at the critical threshold. The spatial (horizontal) and time (vertical) axis show self-similar propagation of diseased individuals in grey, produced by \cite{GRASSBERGER1986273}}
    \label{fig:1d_perc_basis}
\end{figure}

The behaviour of percolation-based epidemic models continued (e.g. \cite{pub.1060474189, pub.1059069981}) to be investigated with various methods such as renormalisation-groups and Monte-Carlo, the critical exponents were found along with phase transition graphs characterising epidemic (super-critical) or extinction (sub-critical) regimes. The properties of both epidemics and forest fire percolation models were studied together in \cite{pub.1052857560}, highlighting their similarity.

The first ecological application came in \cite{pub.1031591030} where forest fire (and epidemic) percolation models were adapted in order to study landscape disturbances. Broadly speaking, landscape disturbances constitute a broad array of physical process which lead to a rapid ecological change, this could include invasion of pests, climate-change and fire.

\textemdash \cite{GRASSBERGER1983157} <- reference for early considerations towards percolation as a model for tree diseases and percolation
\textemdash \cite{SANDER2002293}, read and get more info + research links

\section{\textcolor{red}{Early warning signals}}

\label{section:ews}
\begin{itemize}
    \item Set the scene for chapters 1 and 2, what are early warning signals and how did they develop ?
    \item Review Siro's paper and give examples of how early warning signals have been used to increase tree-health and resilience 
\end{itemize}


\section{\textcolor{red}{Ash dieback}}
% WORK IN PROGRESS

% the combination of ash dieback and emerald ash borer present a significant threat to UK and European ash \cite{musolin2017between}
% managing the spread of ADB in areas already infected is hardly possible \cite{gross2014h}

% - but slowing the spread of disease is still valuable to allow ash populations vital time to recover and policy makers \cite{PAUTASSO201337}

% multi-scale approaches have been outlined \cite{hart2020theoretical}
% multi-seasonal frameworks comprise a common theme in the spread of crop-based epidemics see  x, y, and typically involve soil-borne nematodes-based outbreaks \cite{tankam2020modelling} <- see references inside.
% ash dieback has a strong morality rate \cite{stocks2017first}

\label{ch2:ash-dieback}
Ash dieback (ADB) presents an interesting case-study of an emerging epidemic currently ravaging through large geographic areas of Europe \cite{enderle2019overview}.
The history, and predicted evolution of ash dieback demonstrates how an invasive, non-indigenous pathogen can rapidly spread through a foreign ecosystem which lacks evolutionary defences.
Among the many factors driving the spread of ash dieback, long-distance anthropomorphic trade is the mechanism responsibly for the initial introduction into Europe from the far-east \cite{zhao2013hymenoscyphus, queloz2011cryptic}.

The pathosystem, has been the subject of much research over the years, the taxonomy, symptoms and life-cycle of the pathogen are now well-known \cite{https://doi.org/10.1111/mpp.12073}.
However, the classification and identification of the novel pathogen has an interesting back-story and required a great deal of work.
Understanding the spread of ADB, and managing the epidemic impact to ecosystems, could only be achieved by the confluence of molecular biologists, forest-managers, policy makers and modellers.
Although the epidemic is well underway, and little can be done to contain the spread, slowing the spread of ash dieback remains essential to allow ash population time to adapt.

\subsection{Historical developments}

Initially, a large-scale outbreak of severe ash dieback was first noted in Poland in $1992$, however the causal agent was not identified for a decade \cite{kowalski2001zamieraniu}.
The novel pathogenic fungi was identified as an ascomycete anamorph (i.e. asexual), and given the name \textit{Chalara fraxinea} \cite{kowalski2006chalara}, a member of the hyphomycete genus \textit{Chalara}
The sexual telemorphic stage of the pathogen was later attributed to \textit{Hymenoscyphus albidus} \cite{kowalski2009teleomorph}, a well-known non-pathogenic fungus indigenous to Europe.

Linking the hitherto non-pathogenic \textit{H. albidus} to the casual agent of ash dieback perplexed researchers. 
Although, the enigma was resolved through DNA sequencing by \cite{queloz2011cryptic}, a second morphologically identical ascomycete named \textit{`Hymenoscyphus pseudoalbidus'} was identified as the pathogen responsible for wide-spread dieback of European ash. 

Interestingly, the emergent epidemic caused by \textit{H. pseudoalbidus} coincided with developments in the phylogenetic classification system of the kingdom Fungi \cite{hibbett2007higher}, and dual nomenclature\footnote{Originally, fungi were classified through the structure of their sexual organs. Problematically, ascomycete fungi have a dual `pleomorphic' reproductive mode that often caused confused. However, a move toward a one-name fungi classification system has since simplified fungi taxonomy.} \cite{wingfield2012one}. Subsequently, the pathogen \textit{`Hymenoscyphus pseudoalbidus'} was renamed to \textit{Hymenoscyphus fraxineus}.

The detection of ADB is complicated to its morphologically identical twin \textit{Hymenoscyphus albidus} \cite{queloz2011cryptic} 'Cryptic speciation'

\subsection{Symptoms and epidemiology}

Ash dieback, caused by the fungal pathogen \textit{H. fraxineus} (HF), is a lethal pathogen to European ash (\textit{Fraxinus excelsior}) of all age classes. 
Although, age remains a significant factor of disease severity \cite{marccais2017estimation}.
% talk about other factors of stress...
 
The fungus HF poses little to it's native hosts \textit{Fraxinus mandshurica} and \textit{Fraxinus chinensis}.
However, European ash did not co-evolve with the fungus, and demonstrates a high susceptibility.
Once the fungus colonises a European ash leaf, it can spread through twigs, branches, and the xylem where it eventually spreads throughout the whole tree;
symptoms include necrotic lesions, crown dieback, wilting and eventual death \cite{gross2014h}.

% Leaf-infections are caused by ascospores landing on leaves, and 
% symptoms take around two weeks to develop \cite{khttps://doi.org/10.1111/ppa.12048}.
% Infection through the leafs occurs through....\cite{https://doi.org/10.1111/ppa.12844}

\begin{figure}
    \centering
    \includegraphics[scale=0.5]{chapter2/figures/gross2014.jpg}
    \caption{The symptoms of ash dieback, taken from the work of \cite{gross2014h}. The pathogen \textit{H. fraxineus} infects the leaves of ash, leading to early onset wilting and desiccation. The fungus then reproduces asexually, spreading through twigs, branches and eventually the xylem. Symptoms include wilting, necrotic lesions, crown dieback and eventual mortality.}
    \label{fig:my_label}
\end{figure}
 Once infected, the time ash survive depends on a plethora of factors such as age, surrounding landscape, genetic susceptibility or the particular mode of pathogen infection\footnote{For example, the pathogen can colonise the root-system \cite{schumacher2011general}, usually in severely infected ash \cite{https://doi.org/10.1111/mpp.12073}. From this point, it is only a matter of time before opportunistic fungi invade and significantly accelerate mortality \cite{enderle2013temporal}.}. 
 
 Various sources of ash mortality data have been collected in different European countries. 
 In Germany, a forest stand of planted ash trees showed a $73\%$ mortality rate after five years \cite{langer2015ash} (as cited in a review \cite{enderle2017ash}), while observations of ADB progression in Austria suggest a low mortality rate of $5\%$ measured over a two-year window \cite{kessler2012dieback}. 
 A study conducted at different sites throughout Great Britain suggests a time-scale ranging between $3-15$ years of infected tree growth before death \cite{wylder2018evidence}.

\subsection{Life cycle and reproductive mode}
% talk about reproductive mode
% WIKI: Hymenoscyphus fraxineus has two phases to its life-cycle: sexual and asexual.[9] The asexual stage (anamorph) grows in affected trees attacking the bark and encircling twigs and branches.[9] The sexual, reproductive stage, (teleomorph) grows during summer on ash petioles in the previous year's fallen leaves.[7] The ascospores are produced in asci and are transmitted by wind; this might explain the rapid spread of the fungus.[

The primary natural driver of disease propagation is wind-dispersal during summertime sporulation.
Sporulation typically occurs from June-September when fungal fruiting bodies on the previous years litter-fall release ascospores \cite{grosdidier2018tracking, hietala2013invasive}.
Although the primary natural driver is wind, infection (and re-infection) of ash are also possible through the soil-borne mechanisms \cite{fones2016role}, albeit with low frequency.

Ash infected with the pathogen will then shed its leaves in the autumn. Fungal fruiting bodies then grow on dead leaf litter until summertime; at that point, spores produced by the fruiting body are wind-dispersed and continue the cycle by producing new secondary infections.% expand on the fruiting body spore production.

The reproductive mode of ADB is complicated, leaves can be infected by multiple sources of ascospores \cite{gross2012reproductive}.
Modelling work suggest various environmental factors are important to predict the vulnerability of ash and subsequent spread of disease \cite{dal2014risk, chumanova2019predicting}.

The reproductive mode is complicated and can infect hosts through soil, water and air \cite{gross2012reproductive}.
Ash dieback, caused by the fungus \textit{H.pseudoalbidus} (HP), is a highly seasonal \cite{bengtsson2014seasonal} and follows a complicated, yearly, polycyclic infection cycle.

Specific to ADB, there are two stages of dispersal: the causal agent HP is dispersed on infected leaves during the yearly shed of ash, secondly, wind dispersal of fungal spores. Fortunately, the aggregate behaviour of two-stage dispersal is captured by sampling the amount of fungal spores with distance from a nearby source of infected ash \cite{grosdidier2018tracking}\textemdash discussed more below. 
Be that as it may, an infected tree is treated as the dispersal site in the model, although the realistic description is not so simple.

The life cycle of the fungus HF can be understood to have two well-differentiated reproductive modes, sexual and asexual\textemdash a common trait of phyla Ascomycota, or ascomycetes fungi \cite{hawker2016physiology}.
Asexual reproduction primarily occurs when the pathogen amplifies within an infected host, throughout the year. 
The sexual phase of HP occurs during the summer months, from June until September when wind-dispersed spores infect new ash trees.

The seasonal infection cycle of ADB resembles that of crop-based disease \cite{tankam2020modelling}. Although, in the context of crop disease, removal usually coincides with harvest time. In contrast, ash infected with HP can be removed by the pathogen over a time frame spanning years.

Although research has suggested that asexual reproduction of the pathogen may play a role in driving the pathogen spread \cite{fones2016role}, 
the role of asexual reproduction is still not fully understood.

\subsection{Dispersal}

Fungal spores are known to travel large distances \cite{shaw2006assembling, mundt2009aerial, wingen2013long}.

In their paper, \cite{grosdidier2018tracking} tracked the local and landscape-level dispersal of ascospores produced by \textit{H. fraxineus}.
Two dispersal kernels were used, a thin tailed Gaussian and an inverse power law of the forms:
\begin{equation}
    D(a, r) = \frac{1}{\pi a^2} * \exp\big[-\frac{r^2}{a^2}\big]
    \label{eq:adb-ga}
\end{equation}
and
\begin{equation}
    D(a, r) = \frac{(b-1)(b-2)}{2\pi a^2} * \big[ 1+ \frac{r}{a}\big]^{-b}
    \label{eq:geometric-invserse-power-law}
\end{equation}
where $a$ and $b$ are fitted parameters (For the Gaussian kernel, $a=\sqrt{2}\sigma$ with $\sigma$ being the standard deviation.). Equation \ref{eq:geometric-invserse-power} falls into the classical two-parameter geometric family of dispersal distributions.
The scale parameter is described by $a$ and the shape parameter by $b$.
The mean dispersal distance described by Equation \ref{eq:geometric-invserse-power} is  $\frac{2a}{b-3}$, and parameters $a$ and $b$ are valid for $a>0$ and $b>2$.

The form of Equation \ref{eq:geometric-invserse-power} is based on pollen dispersal studies , as reviewed by \cite{nathan2012dispersal}. 
The particular form of Equation \ref{eq:geometric-invserse-power} is more suited to describe LDD events, as noted by \cite{https://doi.org/10.1111/j.1365-294X.2004.02100.x} in relation to a pollen dispersal. Moreover, a study by \cite{https://doi.org/10.1111/j.1365-294X.2006.03155.x} demonstrated the potential of Equation \ref{eq:geometric-invserse-power} to model pollen-movements (and thus gene-flow) over landscape-level spatial scales. Although pollen is different from ascospores..., ; Presumably, the ability of Equation \ref{eq:geometric-invserse-power} to model LDD events motivated the authors of \cite{grosdidier2018tracking} to include it in their field-study.

% Although data on spore depositions does not necessarily correspond to a new infection, it sheds essential light on the spatial scale of ADB dispersal. 
% draw analogy between exponential kernel of \cite{} <- tracking adb and \cite{} <- landscape spread of adb
% commonly referred to as \textit{H.pseudoalbidus}-\textit{F. excelsior}

\subsection{Management and control}

The control of ash dieback in an established focus of infestation in both natural and artificial ash stands is virtually impossible, as noted by \cite{havrdova2017environmental}. Moreover, it is already well recognised that ADB will wipe out the vast majority of ash, in Europe and Great Britain alike \cite{ash-dieback-costs}. Managing the spread of ADB is a conservation challenge \cite{pautasso2013european} % talk about the control of ADB


% read me ->\cite{MITCHELL201495}

% Approaches to conservation differ between forest and commercial ash stand \cite{skovsgaard2017silvicultural}. 

% Among the control strategies, application of fungicide to infected biomass has been shown effective \cite{hauptman2015application}

\blindtext

\subsection{Modelling}

\blindtext

% statistical modelling work has shown the importance of environmental factors in the spread of disease \cite{dal2014risk, chumanova2019predicting}

% Although, compartmentalised models of ADB have a noticeable absence in the literature.

% the risk of spread.. was conduced by \cite{fellenor2019ash}....

% \cite{dal2015epidemiology} un-published thesis